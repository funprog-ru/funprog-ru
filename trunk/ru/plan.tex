\chapter*{Plan of the lectures}

This chapter indicates roughly how the material is to be distributed over a
course of twelve lectures, each of slightly less than one hour.

\begin{enumerate}

\item {\bf Introduction and Overview} Functional and imperative programming:
contrast, pros and cons. General structure of the course: how lambda calculus
turns out to be a general programming language. Lambda notation: how it
clarifies variable binding and provides a general analysis of mathematical
notation. Currying. Russell's paradox.

\item {\bf Lambda calculus as a formal system} Free and bound variables.
Substitution. Conversion rules. Lambda equality. Extensionality. Reduction and
reduction strategies. The Church-Rosser theorem: statement and consequences.
Combinators.

\item {\bf Lambda calculus as a programming language} Computability background;
Turing completeness (no proof). Representing data and basic operations: truth
values, pairs and tuples, natural numbers. The predecessor operation. Writing
recursive functions: fixed point combinators. Let expressions. Lambda calculus
as a declarative language.

\item {\bf Types} Why types? Answers from programming and logic. Simply typed
lambda calculus. Church and Curry typing. Let polymorphism. Most general types
and Milner's algorithm. Strong normalization (no proof), and its negative
consequences for Turing completeness. Adding a recursion operator.

\item {\bf ML} ML as typed lambda calculus with eager evaluation. Details of
evaluation strategy. The conditional. The ML family. Practicalities of
interacting with ML. Writing functions. Bindings and declarations. Recursive
and polymorphic functions. Comparison of functions.

\item {\bf Details of ML} More about interaction with ML. Loading from files.
Comments. Basic data types: unit, booleans, numbers and strings. Built-in
operations. Concrete syntax and infixes. More examples. Recursive types and
pattern matching. Examples: lists and recursive functions on lists.

\item {\bf Proving programs correct} The correctness problem. Testing and
verification. The limits of verification. Functional programs as mathematical
objects. Examples of program proofs: exponential, GCD, append and reverse.

\item {\bf Effective ML} Using standard combinators. List iteration and other
useful combinators; examples. Tail recursion and accumulators; why tail
recursion is more efficient. Forcing evaluation. Minimizing consing. More
efficient reversal. Use of `as'. Imperative features: exceptions, references,
arrays and sequencing. Imperative features and types; the value restriction.

\item {\bf ML examples I: symbolic differentiation} Symbolic computation. Data
representation. Operator precedence. Association lists. Prettyprinting
expressions. Installing the printer. Differentiation. Simplification. The
problem of the `right' simplification.

\item {\bf ML examples II: recursive descent parsing} Grammars and the parsing
problem. Fixing ambiguity. Recursive descent. Parsers in ML. Parser
combinators; examples. Lexical analysis using the same techniques. A parser for
terms. Automating precedence parsing. Avoiding backtracking. Comparison with
other techniques.

\item {\bf ML examples III: exact real arithmetic} Real numbers and finite
representations. Real numbers as programs or functions. Our representation of
reals. Arbitrary precision integers. Injecting integers into the reals.
Negation and absolute value. Addition; the importance of rounding division.
Multiplication and division by integers. General multiplication. Inverse and
division. Ordering and equality. Testing. Avoiding reevaluation through memo
functions.

\item {\bf ML examples IV: Prolog and theorem proving} Prolog terms.
Case-sensitive lexing. Parsing and printing, including list syntax.
Unification. Backtracking search. Prolog examples. Prolog-style theorem
proving. Manipulating formulas; negation normal form. Basic prover; the use of
continuations. Examples: Pelletier problems and whodunit.

\end{enumerate}

%%% Local Variables:
%%% TeX-master: "all"
%%% End:
