\chapter{Examples}

As we have said, ML was originally designed as a metalanguage for a computer
theorem prover. However it is suitable for plenty of other applications, mostly
from the same general field of `symbolic computation'. In this chapter we will
give some examples of characteristic uses of ML. It is not necessary that the
reader should understand every detail of the particular examples, e.g. the
analyses of real number approximations given below. However it is worth trying
out these programs and similar examples for yourself, and doing some of the
exercises. There is no better way of getting a feel for how ML is actually
used.

\section{Symbolic differentiation}

The phrase `symbolic computation' is a bit vague, but roughly speaking it
covers applications where manipulation of mathematical {\em expressions}, in
general containing variables, is emphasized at the expense of actual numerical
calculation. There are several successful `computer algebra systems' such as
Axiom, Maple and Mathematica, which can do certain symbolic operations that are
useful in mathematics, e.g. factorizing polynomials and differentiating and
integrating expressions. (They are also capable of purely numerical work.) We
will illustrate how this sort of application can be tackled using ML.

We use symbolic differentiation as our example, because there is an algorithm
to do it which is fairly simple and well known. The reader is probably familiar
with certain derivatives of basic operations, e.g ${d \over dx} sin(x) =
cos(x)$, as well as results such as the Chain Rule and Product Rule for
calculating the derivatives of composite expressions in terms of the
derivatives of the components. Just as one can use these systematically in
order to find derivatives, it is possible to program a fairly straightforward
algorithm to do it in ML.

\subsection{First order terms}

We will allow mathematical expressions in a fairly general form. They may be
built up from variables and constants by the application of operators. These
operators may be unary, binary, ternary, or in general $n$-ary. We represent
this using the following recursive datatype:

\begin{boxed}\begin{verbatim}
  #type term = Var of string
             | Const of string
             | Fn of string * (term list);;
  Type term defined.
\end{verbatim}\end{boxed}

\noindent For example, the expression:

$$ sin(x + y) / cos(x - exp(y)) - ln(1 + x) $$

\noindent is represented by:

\begin{boxed}\begin{verbatim}
  Fn("-",[Fn("/",[Fn("sin",[Fn("+",[Var "x"; Var "y"])]);
                  Fn("cos",[Fn("-",[Var "x"; Fn("exp",[Var "y"])])])]);
          Fn("ln",[Fn("+",[Const "1"; Var "x"])])]);;
\end{verbatim}\end{boxed}

\subsection{Printing}

Reading and writing expressions in their raw form is rather unpleasant. This is
a general problem in all symbolic computation systems, and typically the
system's interface contains a {\em parser} and {\em prettyprinter} to allow
respectively input and output in a readable form. A detailed discussion of
parsing is postponed, since it is worth a section of its own, but we will now
write a very simple printer for expressions, so that at least we can see what's
going on. We want the printer to support ordinary human conventions, i.e.

\begin{itemize}

\item Variables and constants are just written as their names.

\item Ordinary $n$-ary functions applied to arguments are written by
juxtaposing the function and a bracketed list of arguments, e.g.
$f(x_1,\ldots,x_n)$.

\item Infix binary functions like $+$ are written in between their
arguments.

\item Brackets are used where necessary for disambiguation.

\item Infix operators have a notion of precedence to reduce the need for
brackets.

\end{itemize}

First we declare a list of infixes, which is a list of pairs: each operator
name together with its precedence.

\begin{boxed}\begin{verbatim}
  #let infixes = ["+",10; "-",10; "*",20; "/",20];;
\end{verbatim}\end{boxed}

It is common to use lists in this way to represent finite partial functions,
since it allows more flexibility. They are commonly called {\em association
lists}, and are very common in functional programming.\footnote{For more
heavyweight applications, some alternative such as hash tables is much better,
but for simple applications where the lists don't get too long, this kind of
linear list is simple and adequate.} In order to convert the list into a
partial function we use {\tt assoc}:

\begin{boxed}\begin{verbatim}
  #let rec assoc a ((x,y)::rest) = if a = x then y else assoc a rest;;
  Toplevel input:
  >let rec assoc a ((x,y)::rest) = if a = x then y else assoc a rest;;
  >        ^^^^^^^^^^^^^^^^^^^^^^^^^^^^^^^^^^^^^^^^^^^^^^^^^^^^^^^^^
  Warning: this matching is not exhaustive.
  assoc : 'a -> ('a * 'b) list -> 'b = <fun>
\end{verbatim}\end{boxed}

The compiler warns us that if the appropriate data is not found in the list,
the function will fail. Now we can define a function to get the infix
precedence of a function:

\begin{boxed}\begin{verbatim}
  #let get_precedence s = assoc s infixes;;
  get_precedence : string -> int = <fun>
  #get_precedence "+";;
  - : int = 10
  #get_precedence "/";;
  - : int = 20
  #get_precedence "%";;
  Uncaught exception: Match_failure ("", 6544, 6601)
\end{verbatim}\end{boxed}

Note that if we ever change this list of infixes, then any functions such as
{\tt get\_precedence} that use it need to be redeclared. This is the main
reason why many LISP programmers prefer dynamic binding. However we can make
the set of infixes arbitrarily extensible by making it into a reference and
dereferencing it on each application:

\begin{boxed}\begin{verbatim}
  #let infixes = ref ["+",10; "-",10; "*",20; "/",20];;
  infixes : (string * int) list ref =
    ref ["+", 10; "-", 10; "*", 20; "/", 20]
  #let get_precedence s = assoc s (!infixes);;
  get_precedence : string -> int = <fun>
  #get_precedence "^";;
  Uncaught exception: Match_failure ("", 6544, 6601)
  #infixes := ("^",30)::(!infixes);;
  - : unit = ()
  #get_precedence "^";;
  - : int = 30
\end{verbatim}\end{boxed}

Note that by the ML evaluation rules, the dereferencing is only applied after
the function {\tt get\_precedence} receives its argument, and hence results in
the set of infixes at that time. We can also define a function to decide
whether a function has any data associated with it. One way is simply to try
the function {\tt get\_precedence} and see if it works:

\begin{boxed}\begin{verbatim}
  #let is_infix s =
     try get_precedence s; true
     with _ -> false;;
\end{verbatim}\end{boxed}

An alternative is to use a general function {\tt can} that finds out whether
another function succeeds:

\begin{boxed}\begin{verbatim}
  #let can f x = try f x; true with _ -> false;;
  can : ('a -> 'b) -> 'a -> bool = <fun>
  #let is_infix = can get_precedence;;
  is_infix : string -> bool = <fun>
\end{verbatim}\end{boxed}

\noindent We will use the following functions that we have already considered:

\begin{boxed}\begin{verbatim}
  #let hd(h::t) = h;;
  #let tl(h::t) = t;;
  #let rec length l = if l = [] then 0 else 1 + length(tl l);;
\end{verbatim}\end{boxed}

\noindent Without further ado, here is a function to convert a term into a
string.

\begin{boxed}\begin{verbatim}
  #let rec string_of_term prec =
     fun (Var s) -> s
       | (Const c) -> c
       | (Fn(f,args)) ->
           if length args = 2 & is_infix f then
             let prec' = get_precedence f in
             let s1 = string_of_term prec' (hd args)
             and s2 = string_of_term prec' (hd(tl args)) in
             let ss = s1^" "^f^" "^s2 in
             if prec' <= prec then "("^ss^")" else ss
           else
             f^"("^(string_of_terms args)^")"

   and string_of_terms t =
     match t with
         [] -> ""
       | [t] -> string_of_term 0 t
       | (h::t) -> (string_of_term 0 h)^","^(string_of_terms t);;
\end{verbatim}\end{boxed}

The first argument {\tt prec} indicates the precedence level of the operator of
which the currently considered expression is an immediate subterm. Now if the
current expression has an infix binary operator at the top level, parentheses
are needed if the precedence of this operator is lower, e.g. if we are
considering the right hand subexpression of $x * (y + z)$. Actually we use
parentheses even when they are equal, in order to make groupings clear. In the
case of associative operators like $+$ this doesn't matter, but we must
distinguish $x - (y - z)$ and $(x - y) - z$. (A more sophisticated scheme would
give each operator an associativity.) The second, mutually recursive, function
{\tt string\_of\_terms} is used to print a comma-separated list of terms, as
they occur for non-infix function applications of the form $f(t_1,\ldots,t_n)$.
Let us see this in action:

\begin{boxed}\begin{verbatim}
  #let t =
    Fn("-",[Fn("/",[Fn("sin",[Fn("+",[Var "x"; Var "y"])]);
                    Fn("cos",[Fn("-",[Var "x";
                                      Fn("exp",[Var "y"])])])]);
            Fn("ln",[Fn("+",[Const "1"; Var "x"])])]);;
  t : term =
   Fn
    ("-",
     [Fn
       ("/",
        [Fn ("sin", [Fn ("+", [Var "x"; Var "y"])]);
         Fn ("cos", [Fn ("-", [Var "x"; Fn ("exp", [Var "y"])])])]);
      Fn ("ln", [Fn ("+", [Const "1"; Var "x"])])])
  #string_of_term 0 t;;
  - : string = "sin(x + y) / cos(x - exp(y)) - ln(1 + x)"
\end{verbatim}\end{boxed}

In fact, we don't need to convert a term to a string ourselves. CAML Light
allows us to set up our own printer so that it is used to print anything of
type {\tt term} in the standard read-eval-print loop. This needs a few special
commands to ensure that we interact properly with the toplevel
loop:

\begin{boxed}\begin{verbatim}
  ##open "format";;
  #let print_term s =
     open_hvbox 0;
     print_string("`"^(string_of_term 0 s)^"`");
     close_box();;
  print_term : term -> unit = <fun>
  #install_printer "print_term";;
  - : unit = ()
\end{verbatim}\end{boxed}

\noindent Now compare the effect:

\begin{boxed}\begin{verbatim}
  #let t =
      Fn("-",[Fn("/",[Fn("sin",[Fn("+",[Var "x"; Var "y"])]);
                      Fn("cos",[Fn("-",[Var "x";
                                        Fn("exp",[Var "y"])])])]);
              Fn("ln",[Fn("+",[Const "1"; Var "x"])])]);;
  t : term = `sin(x + y) / cos(x - exp(y)) - ln(1 + x)`
  #let x = t
  x : term = `sin(x + y) / cos(x - exp(y)) - ln(1 + x)`
\end{verbatim}\end{boxed}

Once the new printer is installed, it is used whenever CAML wants to print
something of type {\tt term}, even if it is a composite datastructure
such as a pair:

\begin{boxed}\begin{verbatim}
  #(x,t);;
  - : term * term =
   `sin(x + y) / cos(x - exp(y)) - ln(1 + x)`,
   `sin(x + y) / cos(x - exp(y)) - ln(1 + x)`
\end{verbatim}\end{boxed}

\noindent or a list:

\begin{boxed}\begin{verbatim}
  #[x; t; x];;
  - : term list =
   [`sin(x + y) / cos(x - exp(y)) - ln(1 + x)`;
    `sin(x + y) / cos(x - exp(y)) - ln(1 + x)`;
    `sin(x + y) / cos(x - exp(y)) - ln(1 + x)`]
\end{verbatim}\end{boxed}

This printer is rather crude in that it will not break up large expressions
intelligently across lines. The CAML library {\tt format}, from which we used a
few functions above, provides for a better approach. Instead of converting the
term into a string then printing it in one piece, we can make separate printing
calls for different parts of it, and intersperse these with some special
function calls that help the printing mechanism to understand where to break
the lines. Many of the principles used in this and similar `prettyprinting
engines' are due to \citeN{oppen-pretty}. We will not consider this in detail
here --- see the CAML manual for details.\footnote{There is also a tutorial,
written by Pierre Weis, available online as part of the CAML page: `{\tt
http://pauillac.inria.fr/caml/FAQ/format-eng.html}'.}

\subsection{Derivatives}

Now it is a fairly simple matter to define the derivative function. First let
us recall the systematic method we are taught in school:

\begin{itemize}

\item If the expression is one of the standard functions applied to an argument
that is the variable of differentiation, e.g. $sin(x)$, return the known
derivative.

\item If the expression is of the form $f(x) + g(x)$ then apply the rule
for sums, returning $f'(x) + g'(x)$. Likewise for subtraction etc.

\item If the expression is of the form $f(x) * g(x)$ then apply the
product rule, i.e. return $f'(x) * g(x) + f(x) * g'(x)$.

\item If the expression is one of the standard functions applied to a composite
argument, say $f(g(x))$ then apply the Chain Rule and so give
$g'(x) * f'(g(x))$

\end{itemize}

This is nothing but a recursive algorithm, though that terminology is seldom
used in schools. We can program it in ML very directly:

\begin{boxed}\begin{verbatim}
  #let rec differentiate x tm = match tm with
       Var y -> if y = x then Const "1" else Const "0"
     | Const c -> Const "0"
     | Fn("-",[t]) -> Fn("-",[differentiate x t])
     | Fn("+",[t1;t2]) -> Fn("+",[differentiate x t1;
                                   differentiate x t2])
     | Fn("-",[t1;t2]) -> Fn("-",[differentiate x t1;
                                   differentiate x t2])
     | Fn("*",[t1;t2]) ->
        Fn("+",[Fn("*",[differentiate x t1; t2]);
                Fn("*",[t1; differentiate x t2])])
     | Fn("inv",[t]) -> chain x t
        (Fn("-",[Fn("inv",[Fn("^",[t;Const "2"])])]))
     | Fn("^",[t;n]) -> chain x t
       (Fn("*",[n; Fn("^",[t; Fn("-",[n; Const "1"])])]))
     | Fn("exp",[t]) -> chain x t tm
     | Fn("ln",[t]) -> chain x t (Fn("inv",[t]))
     | Fn("sin",[t]) -> chain x t (Fn("cos",[t]))
     | Fn("cos",[t]) -> chain x t
        (Fn("-",[Fn("sin",[t])]))
     | Fn("/",[t1;t2]) -> differentiate x
        (Fn("*",[t1; Fn("inv",[t2])]))
     | Fn("tan",[t]) -> differentiate x
        (Fn("/",[Fn("sin",[t]); Fn("cos",[t])]))
   and chain x t u =  Fn("*",[differentiate x t; u]);;
\end{verbatim}\end{boxed}

The {\tt chain} function just avoids repeating the chain rule construction for
many operations. Apart from that, we just systematically apply rules for sums,
products etc. and the known derivatives of standard functions. Of course, we
could add other functions such as the hyperbolic functions and inverse
trigonometric functions if desired. A couple of cases, namely division and
$tan$, avoid a complicated definition by applying the differentiator to an
alternative formulation.

\subsection{Simplification}

If we try the differentiator out on our running example, then it works quite
well:

\begin{boxed}\begin{verbatim}
  #t;;
  - : term = `sin(x + y) / cos(x - exp(y)) - ln(1 + x)`
  #differentiate "x" t;;
  - : term =
   `(((1 + 0) * cos(x + y)) * inv(cos(x - exp(y))) +
    sin(x + y) * (((1 - 0 * exp(y)) * -(sin(x - exp(y)))) *
    -(inv(cos(x - exp(y)) ^ 2)))) - (0 + 1) * inv(1 + x)`
  #differentiate "y" t;;
  - : term =
   `(((0 + 1) * cos(x + y)) * inv(cos(x - exp(y))) +
    sin(x + y) * (((0 - 1 * exp(y)) * -(sin(x - exp(y)))) *
    -(inv(cos(x - exp(y)) ^ 2)))) - (0 + 0) * inv(1 + x)`
  #differentiate "z" t;;
  - : term =
   `(((0 + 0) * cos(x + y)) * inv(cos(x - exp(y))) +
    sin(x + y) * (((0 - 0 * exp(y)) * -(sin(x - exp(y)))) *
    -(inv(cos(x - exp(y)) ^ 2)))) - (0 + 0) * inv(1 + x)`
\end{verbatim}\end{boxed}

However, it fails to make various obvious simplifications such as $0 * x = 0$
and $x + 0 = x$. Some of these redundant expressions are created by the rather
unsubtle differentiation strategy which, for instance, applies the chain rule
to $f(t)$ even where $t$ is just the variable of differentiation. However, some
would be hard to avoid without making the differentiator much more complicated.
Accordingly, let us define a separate simplification function that gets rid of
some of these unnecessary expressions.

\begin{boxed}\begin{verbatim}
  #let simp =
     fun (Fn("+",[Const "0"; t])) -> t
       | (Fn("+",[t; Const "0"])) -> t
       | (Fn("-",[t; Const "0"])) -> t
       | (Fn("-",[Const "0"; t])) -> Fn("-",[t])
       | (Fn("+",[t1; Fn("-",[t2])])) -> Fn("-",[t1; t2])
       | (Fn("*",[Const "0"; t])) -> Const "0"
       | (Fn("*",[t; Const "0"])) -> Const "0"
       | (Fn("*",[Const "1"; t])) -> t
       | (Fn("*",[t; Const "1"])) -> t
       | (Fn("*",[Fn("-",[t1]); Fn("-",[t2])])) -> Fn("*",[t1; t2])
       | (Fn("*",[Fn("-",[t1]); t2])) -> Fn("-",[Fn("*",[t1; t2])])
       | (Fn("*",[t1; Fn("-",[t2])])) -> Fn("-",[Fn("*",[t1; t2])])
       | (Fn("-",[Fn("-",[t])])) -> t
       | t -> t;;
\end{verbatim}\end{boxed}

This just applies the simplification at the top level of the term. We need to
execute it in a bottom-up sweep. This will also bring negations upwards through
products. In some cases it would be more efficient to simplify in a top-down
sweep, e.g. $0 * t$ for a complicated expression $t$, but this would need
repeating for terms like $(0 + 0) * 2$. The choice is rather like that between
normal and applicative order reduction in lambda calculus.

\begin{boxed}\begin{verbatim}
  #let rec dsimp =
     fun (Fn(fn,args)) -> simp(Fn(fn,map dsimp args))
       | t -> simp t;;
\end{verbatim}\end{boxed}

\noindent Now we get better results:

\begin{boxed}\begin{verbatim}
  #dsimp(differentiate "x" t);;
  - : term =
   `(cos(x + y) * inv(cos(x - exp(y))) +
    sin(x + y) * (sin(x - exp(y)) *
    inv(cos(x - exp(y)) ^ 2))) - inv(1 + x)`
  #dsimp(differentiate "y" t);;
  - : term =
   `cos(x + y) * inv(cos(x - exp(y))) -
    sin(x + y) * ((exp(y) * sin(x - exp(y))) *
    inv(cos(x - exp(y)) ^ 2))`
  #dsimp(differentiate "z" t);;
  - : term = `0`
\end{verbatim}\end{boxed}

In general, one can always add more and more sophisticated simplification. For
example, consider:

\begin{boxed}\begin{verbatim}
  #let t2 = Fn("tan",[Var "x"]);;
  t2 : term = `tan(x)`
  #differentiate "x" t2;;
  - : term =
   `(1 * cos(x)) * inv(cos(x)) +
    sin(x) * ((1 * -(sin(x))) * -(inv(cos(x) ^ 2)))`
  #dsimp(differentiate "x" t2);;
  - : term = `cos(x) * inv(cos(x)) +
              sin(x) * (sin(x) * inv(cos(x) ^ 2))`
\end{verbatim}\end{boxed}

We would like to simplify {\tt cos(x) * inv(cos(x))} simply to {\tt 1},
ignoring, as most commercial computer algebra systems do, the possibility that
{\tt cos(x)} might be zero. We can certainly add new clauses to the simplifier
for that. Similarly, we might like to collect up terms in the second summand.
However here we start to see that human psychology plays a role. Which of the
following is preferable? It is hard for the machine to decide unaided.

\begin{boxed}\begin{verbatim}
  sin(x) * (sin(x) * inv(cos(x) ^ 2))

  sin(x) ^ 2 * inv(cos(x) ^ 2)

  sin(x) ^ 2 / cos(x) ^ 2

  (sin(x) / cos(x)) ^ 2

  tan(x) ^ 2
\end{verbatim}\end{boxed}

And having chosen the last one, should it then convert

\begin{boxed}\begin{verbatim}
  1 + tan(x) ^ 2
\end{verbatim}\end{boxed}

\noindent into

\begin{boxed}\begin{verbatim}
  sec(x) ^ 2
\end{verbatim}\end{boxed}

Certainly, it is shorter and more conventional. However it relies on the fact
that we know the definitions of these fairly obscure trig functions like {\tt
sec}. Whatever the machine decides to do, it is likely that some user will find
it upsetting.

\section{Parsing}

A defect of our previous example was that the input had to be written
explicitly in terms of type constructors. Here we will show how to write a
parser for this situation, and make our code sufficiently general that it can
easily be applied to similar formal languages. Generally speaking, the problem
of parsing is as follows. A formal language is defined by a {\em grammar},
which is a set of {\em production rules} for each syntactic category. For a
language of terms with two infix operators {\tt +} and {\tt *} and only
alphanumeric variables and numeral ($0$, $1$ etc.) constants, it might look
like this:

\begin{eqnarray*}
term     & \goesto & name{\verb!(!} termlist {\verb!)!}         \\
         & |       & name                                       \\
         & |       & {\verb!(!} term {\verb!)!}                 \\
         & |       & numeral                                    \\
         & |       & {\verb!-!} term                            \\
         & |       & term {\verb! + !} term                     \\
         & |       & term {\verb! * !} term                     \\
termlist & \goesto & term {\verb!,!} termlist                   \\
         & |       & term
\end{eqnarray*}

We will assume for the moment that we know what names and numerals are, but we
could have similar rules defining them in terms of single characters. Now this
set of production rules gives us a way of generating the concrete, linear
representation for all terms. Note that $name$, $numeral$, $term$ and
$termlist$ are meant to be variables in the above, whereas elements like
`{\verb!+!}' and `{\verb!(!}' written in bold type are actual characters. For
example we have:

\begin{eqnarray*}
term     & \goesto & term {\verb! * !} term                             \\
         & \goesto & term {\verb! * !} {\verb!(!} term {\verb!)!}       \\
         & \goesto & term {\verb! * !} {\verb!(!}
                                term {\verb! + !} term {\verb!)!}       \\
         & \goesto & term {\verb! * !} {\verb!(!}
                     term {\verb! + !} term {\verb! * !} term {\verb!)!}\\
         & \goesto & numeral {\verb! * !} {\verb!(!}
                     name {\verb! + !} name {\verb! * !} name {\verb!)!}
\end{eqnarray*}


\noindent so, following similar rules for $name$ and $numeral$, we can
generate for example:

$$ {\verb!10 * (x + y * z)!} $$

The task of {\em parsing} is to reverse this process of applying production
rules, i.e. to take a string and discover how it could have been generated by
the production rules for a syntactic category. Typically, the output is a {\em
parse tree}, i.e. a tree structure that shows clearly the sequence of
applications of the rules.

One problem with parsing is that the grammar may be {\em ambiguous}, i.e. a
given string can often be generated in several different ways. This is the case
in our example above, for the string could also have been generated by:

\begin{eqnarray*}
term     & \goesto & term {\verb! * !} term                             \\
         & \goesto & term {\verb! * !} {\verb!(!} term {\verb!)!}       \\
         & \goesto & term {\verb! * !} {\verb!(!} term {\verb! * !}
                     term {\verb!)!}                                    \\
         & \goesto & term {\verb! * !} {\verb!(!}
                     term {\verb! + !} term {\verb! * !} term {\verb!)!}\\
         & \goesto & numeral {\verb! * !} {\verb!(!}
                     name {\verb! + !} name {\verb! * !} name {\verb!)!}
\end{eqnarray*}

The obvious way to fix this is to specify rules of precedence and associativity
for infix operators. However, we can make the grammar unambiguous without
introducing additional mechanisms at the cost of adding new syntactic
categories:

\begin{eqnarray*}
atom     & \goesto & name{\verb!(!} termlist {\verb!)!}         \\
         & |       & name                                       \\
         & |       & numeral                                    \\
         & |       & {\verb!(!} term {\verb!)!}                 \\
         & |       & {\verb!-!} atom                            \\
mulexp   & \goesto & atom {\verb! * !} mulexp                   \\
         & |       & atom                                       \\
term     & \goesto & mulexp {\verb! + !} term                   \\
         & |       & mulexp                                     \\
termlist & \goesto & term {\verb!,!} termlist                   \\
         & |       & term
\end{eqnarray*}

Note that this makes both infix operators right associative. For a more
complicated example of this technique in action, look at the formal definition
of expressions in the ANSI C Standard.

\subsection{Recursive descent parsing}

The production rules suggest a very simple way of going about parsing using a
series of mutually recursive functions. The idea is to have a function for each
syntactic category, and a recursive structure that reflects exactly the mutual
recursion found in the grammar itself. For example, the procedure for parsing
terms, say {\verb!term!} will, on encountering a {\verb!-!} symbol, make a
recursive call to itself to parse the subterm, and on encountering a name
followed by an opening parenthesis, will make a recursive call to
{\verb!termlist!}. This in itself will make at least one recursive call to
{\verb!term!}, and so on. Such a style of programming is particularly natural
in a language like ML where recursion is the principal control mechanism.

In our ML implementation, we suppose that we have some input list of objects of
arbitrary type $\alpha$ for the parser to consume. These might simply be
characters, but in fact there is usually a separate level of {\em lexical
analysis} which collects characters together into {\em tokens} such as
`{\verb!x12!}', `{\verb!:=!}' and `{\verb!96!}', so by the time we reach this
level the input is a list of tokens. We avoid committing ourselves to any
particular type. Similarly, we will allow the parser to construct objects of
arbitrary type $\beta$ from its input. These might for example be parse trees
as members of a recursive datatype, or might simply be numbers if the parser is
to take an expression and evaluate it. Now in general, a parser might not
consume all its input, so we also make it output the remaining tokens.
Therefore, a parser will have type

$$ (\alpha)list \to \beta \times (\alpha)list $$

For example, when given the input characters {\verb!(x + y) * z!} the
function {\verb!atom!} should process the characters {\verb!(x + y)!} and leave
the remaining characters {\verb!* z!}. It might return a parse tree for the
processed expression using our earlier recursive type, and hence we would have:

\begin{boxed}\begin{lstlisting}
  atom "(x + y) * z" = Fn("+",[Var "x"; Var "y"]),"* z"
\end{lstlisting}\end{boxed}

Since any call of the function {\verb+atom+} must be from within the function
{\verb!mulexp!}, this second function will use the result of {\verb!atom!} and
deal with the input that remains, making a second call to {\verb!atom!} to
process the expression `{\verb!z!}'.

\subsection{Parser combinators}

Another reason why this technique works particularly well in ML is that we can
define some useful combinators for plugging parsers together. In fact, by
giving them infix status, we can make the ML parser program look quite similar
in structure to the original grammar defining the language.

First we declare an exception to be used where parsing fails. Then we define an
infix {\verb!++!} that applies two parsers in sequence, pairing up their
results, and and infix {\verb!||!} which tries first one parser, then the
other. We then define a many-fold version of {\verb!++!} that repeats a parser
as far as possible, putting the results into a list. Finally, the infix
{\verb!>>!} is used to modify the results of a parser according to a function.

The CAML rules for symbolic identifiers such as {\verb!++!} make them infix
automatically, so we suppress this during the definition using the {\tt prefix}
keyword. Their precedences are also automatic, following from the usual
precedence of their first characters as arithmetic operations etc. That is,
{\verb!++!} is the strongest, {\verb!>>!} falls in the middle, and {\verb!||!}
is the weakest; this is exactly what we want.

\begin{boxed}\begin{lstlisting}
  exception Noparse;;

  let prefix || parser1 parser2 input =
    try parser1 input
    with Noparse -> parser2 input;;

  let prefix ++ parser1 parser2 input =
    let result1,rest1 = parser1 input in
    let result2,rest2 = parser2 rest1 in
    (result1,result2),rest2;;

  let rec many parser input =
    try let result,next = parser input in
        let results,rest = many parser next in
        (result::results),rest
    with Noparse -> [],input;;

  let prefix >> parser treatment input =
    let result,rest = parser input in
    treatment(result),rest;;
\end{lstlisting}\end{boxed}

We will, in what follows, use the following other functions. Most of these
have been defined above, the main exception being {\tt explode} which converts
a string into a list of 1-elements strings. It uses the inbuilt functions
{\verb!sub_string!} and {\verb!string_length!}; we do not describe them in
detail but their function should be easy to grasp from the example.

\begin{boxed}\begin{verbatim}
  let rec itlist f =
       fun [] b -> b
         | (h::t) b -> f h (itlist f t b);;

  let uncurry f(x,y) = f x y;;

  let K x y = x;;

  let C f x y = f y x;;

  let o f g x = f(g x);;
  #infix "o";;

  let explode s =
    let rec exap n l =
        if n < 0 then l else
        exap (n - 1) ((sub_string s n 1)::l) in
    exap (string_length s - 1) [];;
\end{verbatim}\end{boxed}

In order to get started, we define a few `atomic' parsers. The function {\tt
some} accepts any item satisfying a predicate and returns it. The function {\tt
a} is similar, but demands a particular item, and finally {\tt finished} just
makes sure there are no items left to be parsed.

\begin{boxed}\begin{lstlisting}
  let some p =
    fun [] -> raise Noparse
      | (h::t) -> if p h then (h,t) else raise Noparse;;

  let a tok = some (fun item -> item = tok);;

  let finished input =
    if input = [] then 0,input else raise Noparse;;
\end{lstlisting}\end{boxed}

\subsection{Lexical analysis}

Our parsing combinators can easily be used, in conjunction with a few simple
character discrimination functions, to construct a lexical analyzer for our
language of terms. We define a type of {\em tokens} (or lexemes), and the lexer
translates the input string into a list of tokens. The `other' cases cover
symbolic identifiers; they are all straightforward as they consist of just a
single character, rather than composite ones like `{\verb!:=!}'.

\begin{boxed}\begin{lstlisting}
  type token = Name of string | Num of string | Other of string;;

  let lex =
    let several p = many (some p) in
    let lowercase_letter s = "a" <= s & s <= "z" in
    let uppercase_letter s = "A" <= s & s <= "Z" in
    let letter s = lowercase_letter s or uppercase_letter s in
    let alpha s = letter s or s = "_" or s = "'" in
    let digit s = "0" <= s & s <= "9" in
    let alphanum s = alpha s or digit s in
    let space s = s = " " or s = "\n" or s = "\t" in
    let collect(h,t) = h^(itlist (prefix ^) t "") in
    let rawname =
       some alpha ++ several alphanum >> (Name o collect) in
    let rawnumeral =
       some digit ++ several digit >> (Num o collect) in
    let rawother = some (K true) >> Other in
    let token =
      (rawname || rawnumeral || rawother) ++ several space >> fst in
    let tokens = (several space ++ many token) >> snd in
    let alltokens = (tokens ++ finished) >> fst in
    fst o alltokens o explode;;
\end{lstlisting}\end{boxed}

\noindent For example:

\begin{boxed}\begin{verbatim}
  #lex "sin(x + y) * cos(2 * x + y)";;
   - : token list =
   [Name "sin"; Other "("; Name "x"; Other "+"; Name "y"; Other ")";
    Other "*"; Name "cos"; Other "("; Num "2"; Other "*"; Name "x";
    Other "+"; Name "y"; Other ")"]
\end{verbatim}\end{boxed}

\subsection{Parsing terms}

Since we are now working at the level of tokens, we define trivial parsers to
accept tokens of (only) a certain kind:

\begin{boxed}\begin{lstlisting}
  let name =
    fun (Name s::rest) -> s,rest
      | _ -> raise Noparse;;

  let numeral =
    fun (Num s::rest) -> s,rest
      | _ -> raise Noparse;;

  let other =
    fun (Other s::rest) -> s,rest
      | _ -> raise Noparse;;
\end{lstlisting}\end{boxed}

\noindent Now we can define a parser for terms, in a form very similar to the
original grammar. The main difference is that each production rule has
associated with it some sort of special action to take as a result of parsing.

\begin{boxed}\begin{lstlisting}
  let rec atom input
         = (name ++ a (Other "(") ++ termlist ++ a (Other ")")
                >> (fun (((name,_),args),_) -> Fn(name,args))
         || name
                >> (fun s -> Var s)
         || numeral
                >> (fun s -> Const s)
         || a (Other "(") ++ term ++ a (Other ")")
                >> (snd o fst)
         || a (Other "-") ++ atom
                >> snd) input
    and mulexp input
         = (atom ++ a(Other "*") ++ mulexp
                >> (fun ((a,_),m) -> Fn("*",[a;m]))
         || atom) input
    and term input
         = (mulexp ++ a(Other "+") ++ term
                >> (fun ((a,_),m) -> Fn("+",[a;m]))
         || mulexp) input
    and termlist input
         = (term ++ a (Other ",") ++ termlist
                >> (fun ((h,_),t) -> h::t)
         || term
                >> (fun h -> [h])) input;;
\end{lstlisting}\end{boxed}

\noindent Let us package everything up as a single parsing function:

\begin{boxed}\begin{lstlisting}
  let parser = fst o (term ++ finished >> fst) o lex;;
\end{lstlisting}\end{boxed}

\noindent To see it in action, we try with and without the printer (see above)
installed:

\begin{boxed}\begin{verbatim}
  #parser "sin(x + y) * cos(2 * x + y)";;
  - : term =
   Fn
    ("*",
     [Fn ("sin", [Fn ("+", [Var "x"; Var "y"])]);
      Fn ("cos", [Fn ("+", [Fn ("*", [Const "2"; Var "x"]);
                            Var "y"])])])
  #install_printer "print_term";;
  - : unit = ()
  #parser "sin(x + y) * cos(2 * x + y)";;
  - : term = `sin(x + y) * cos(2 * x + y)`
\end{verbatim}\end{boxed}

\subsection{Automatic precedence parsing}

In the above parser, we `hardwired' a parser for the two infix operators.
However it would be nicer, and consistent with our approach to printing, if
changes to the set of infixes could propagate to the parser. Moreover, even
with just two different binary operators, our production rules and consequent
parser were already starting to look artificial; imagine if we had a dozen or
so. What we would like is to automate the production of a layer of parsers, one
for each operator, in order of precedence. This can easily be done. First, we
define a general function that takes a parser for what, at this level, are
regarded as atomic expressions, and produces a new parser for such expressions
separated by a given operator. Note that this could also be defined using
parsing combinators, but the function below is simple and more efficient.

\begin{boxed}\begin{lstlisting}
  let rec binop op parser input =
    let atom1,rest1 as result = parser input in
    if not rest1 = [] & hd rest1 = Other op then
      let atom2,rest2 = binop op parser (tl rest1) in
      Fn(op,[atom1; atom2]),rest2
    else result;;
\end{lstlisting}\end{boxed}

Now we define functions that pick out the infix operator in our association
list with (equal) lowest precedence, and delete it from the list. If we
maintained the list already sorted in precedence order, we could simply take
the head and tail of the list.

\begin{boxed}\begin{lstlisting}
  let findmin l = itlist
    (fun (_,pr1 as p1) (_,pr2 as p2) -> if pr1 <= pr2 then p1 else p2)
    (tl l) (hd l);;

  let rec delete x (h::t) = if h = x then t else h::(delete x t);;
\end{lstlisting}\end{boxed}

\noindent Now we can define the generic precedence parser:

\begin{boxed}\begin{lstlisting}
  let rec precedence ilist parser input =
    if ilist = [] then parser input else
    let opp = findmin ilist in
    let ilist' = delete opp ilist in
    binop (fst opp) (precedence ilist' parser) input;;
\end{lstlisting}\end{boxed}

\noindent and hence can modify the main parser to be both simpler and more
general:

\begin{boxed}\begin{lstlisting}
  let rec atom input
         = (name ++ a (Other "(") ++ termlist ++ a (Other ")")
                >> (fun (((name,_),args),_) -> Fn(name,args))
         || name
                >> (fun s -> Var s)
         || numeral
                >> (fun s -> Const s)
         || a (Other "(") ++ term ++ a (Other ")")
                >> (snd o fst)
         || a (Other "-") ++ atom
                >> snd) input
    and term input = precedence (!infixes) atom input
    and termlist input
         = (term ++ a (Other ",") ++ termlist
                >> (fun ((h,_),t) -> h::t)
         || term
                >> (fun h -> [h])) input;;

  let parser = fst o (term ++ finished >> fst) o lex;;
\end{lstlisting}\end{boxed}

\noindent for example:

\begin{boxed}\begin{verbatim}
  #parser "2 * sin(x)^2 + 2 * sin(y)^2 - 2";;
  - : term =
   Fn
    ("+",
     [Fn ("*", [Const "2"; Fn ("^", [Fn ("sin", [Var "x"]);
                                     Const "2"])]);
      Fn
       ("-",
        [Fn ("*", [Const "2"; Fn ("^", [Fn ("sin", [Var "y"]);
                                        Const "2"])]);
         Const "2"])])
  #install_printer "print_term";;
  - : unit = ()
  #parser "2 * sin(x)^2 + 2 * sin(y)^2 - 2";;
  - : term = `2 * sin(x) ^ 2 + (2 * sin(y) ^ 2 - 2)`
\end{verbatim}\end{boxed}

\subsection{Defects of our approach}

Our approach to lexing and parsing is not particularly efficient. In fact,
CAML and some other ML implementations include a generator for LR parsers
similar to YACC (Yet Another Compiler Compiler) which is often used to generate
parsers in C under Unix. Not only are these more general in the grammars that
they can handle, but the resulting parsers are more efficient. However we
believe the present approach is rather clear, adequate for most purposes, and a
good introduction to programming with higher order functions.

The inefficiency of our approach can be reduced by avoiding a few particularly
wasteful features. Note that if two different productions for the same
syntactic category have a common prefix, then we should avoid parsing it more
than once, unless it is guaranteed to be trivial. Our production rules for
$term$ have this property:

\begin{eqnarray*}
term     & \goesto & name{\verb!(!} termlist {\verb!)!}         \\
         & |       & name                                       \\
         & |       & \cdots
\end{eqnarray*}

We carefully put the longer production first in our actual implementation,
otherwise success in reading a name would cause the abandonment of attempts to
read a parenthesized list of arguments. Nevertheless, because of the
redundancy we have mentioned, this way of constructing the parser is wasteful.
In this case it doesn't matter much, because the duplicated call only analyzes
one token. However the case for $termlist$ is more significant:

\begin{boxed}\begin{lstlisting}
    let ...
    and termlist input
         = (term ++ a (Other ",") ++ termlist
                >> (fun ((h,_),t) -> h::t)
         || term
                >> (fun h -> [h])) input;;
\end{lstlisting}\end{boxed}

Here we could reparse a whole term, which might be arbitrarily large. We can
avoid this in several ways. One is to move the {\verb!||!} alternation to after
the initial term has been read. A convenient way to code it is to eschew
explicit recursion in favour of {\tt many}:

\begin{boxed}\begin{lstlisting}
    let ...
    and termlist input
        = term ++ many (a (Other ",") ++ term >> snd)
                >> (fun (h,t) -> h::t) input;;
\end{lstlisting}\end{boxed}

\noindent Therefore the final version of the parser is:

\begin{boxed}\begin{lstlisting}
  let rec atom input
         = (name ++ a (Other "(") ++ termlist ++ a (Other ")")
                >> (fun (((name,_),args),_) -> Fn(name,args))
         || name
                >> (fun s -> Var s)
         || numeral
                >> (fun s -> Const s)
         || a (Other "(") ++ term ++ a (Other ")")
                >> (snd o fst)
         || a (Other "-") ++ atom
                >> snd) input
    and term input = precedence (!infixes) atom input
    and termlist input
          = (term ++ many (a (Other ",") ++ term >> snd)
                >> (fun (h,t) -> h::t)) input;;
\end{lstlisting}\end{boxed}

A general defect of our approach, and recursive descent parsing generally, is
that so-called {\em left recursion} in productions causes problems. This is
where a given category expands to a sequence including the category itself as
the first item. For example, if we had wanted to make the addition operator
left-associative in our earlier grammar, we could have used:

\begin{eqnarray*}
term     & \goesto & term {\verb! + !} mulexp                   \\
         & |       & mulexp
\end{eqnarray*}

The naive transcription into ML would loop indefinitely, calling {\tt term} on
the same subterm over and over. There are various slightly ad hoc ways of
dealing with this problem. For example, one can convert this use of recursion,
which in some ways is rather gratuitous, into the explicit use of repetition.
We can do this using standard combinators like {\tt many} to return a list of
`$mulexp$'s and then run over the list building the tree left-associated.

A final problem is that we do not handle errors very gracefully. We always
raise the exception {\tt Noparse}, and this, when caught, initiates
backtracking. However this is not always appropriate for typical grammars. At
best, it can cause expensive reparsing, and can lead to baffling error
indications at the top level. For example, on encountering {\tt 5 +}, we expect
a term following the {\tt +}, and if we don't find one, we should generate a
comprehensible exception for the user, rather than raise {\tt Noparse} and
perhaps initiate several other pointless attempts to parse the expression in a
different way.

\section{Exact real arithmetic}

Real arithmetic on computers is normally done via floating point
approximations. In general, we can only manipulate a real number, either
ourselves or inside a computer, via some sort of finite representation. Some
question how numbers can be said to `exist' if they have no finite
representation. For example, Kronecker accepted integers and rationals because
they can be written down explicitly, and even {\em algebraic}
numbers\footnote{Algebraic numbers are those that are roots of polynomials with
integer coefficients, e.g. $\sqrt{2}$, and transcendental numbers are ones that
aren't.} because they can be represented using the polynomials of which they
are solutions. However he rejected transcendental numbers because apparently
they could not be represented finitely. Allegedly he he greeted the famous
proof by \citeN{lindemann-pi} that $\pi$ is transcendental with the remark that
this was `interesting, except that $\pi$ does not exist'.

However, given our modern perspective, we can say that after all many more
numbers than Kronecker would have accepted {\em do} have a finite
representation, namely the program, or in some more abstract sense the {\em
rule}, used to calculate them to greater and greater precision. For example, we
can write a program that will produce for any argument $n$ the first $n$ digits
of $\pi$. Alternatively it can produce a rational number $r$ such that $|\pi -
r| < 2^{-n}$. Whatever approach is taken to the successive approximation of a
real number, the key point is that its representation, the program itself, is
finite.

This works particularly nicely in a language like ML where higher order
functions are available. What we have called a `program' above will just be an
ML function. We can actually represent the arithmetic operations on numbers as
higher order functions that, given functions for approximating $x$ and $y$,
will produce new ones for approximating $x + y$, $x y$, $sin(x)$ and so on, for
a wide range of functions. In an ordinary programming language, we would need
to define a concrete representation for programs, e.g. `G\"odel numbering', and
write an interpreter for this representation.\footnote{Effectively, we are
manipulating functions `in extension' rather than `in intension', i.e. not
concerning ourselves with their representation.}

\subsection{Representation of real numbers}

Our approach is to represent a real $x$ by a function $f_x:\nat \to \num$ that
for each $n \in \nat$ returns an approximation of $x$ to within $2^{-n}$,
appropriately scaled. In fact, we have:

$$ |f_x(n) - 2^n x| < 1 $$

This is of course equivalent to $|{f_x(n) \over 2^n} - x| < {1 \over 2^n}$. We
could return a rational approximation directly, but it would still be
convenient for the rationals to have denominators that are all powers of some
number, so that common denominators arising during addition don't get too big.
Performing the scaling directly seems simpler, requiring only integer
arithmetic. Our choice of the base $2$ is largely arbitrary. A lower base is
normally advantageous because it minimizes the {\em granularity} of the
achievable accuracies. For example, if we used base 10, then even if we only
need to increase the accuracy of an approximation slightly, we have to step up
the requested accuracy by a factor of 10.

\subsection{Arbitrary-precision integers}

CAML's standard integers (type {\tt int}) have a severely limited range, so
first of all we need to set up a type of unlimited-precision integers. This is
not so difficult to program, but is slightly tedious. Fortunately, the CAML
release includes a library that gives a fast implementation of arbitrary
precision integer (and in fact rational) arithmetic. A version of CAML Light
with this library pre-loaded is installed on Thor. Assuming you have used the
following path setting:

\begin{boxed}\begin{lstlisting}[language=sh]
  PATH="$PATH:/home/jrh13/caml/bin"
  export PATH
\end{lstlisting}\end{boxed}

\noindent then the augmented version of CAML Light can be fired up using:

\begin{boxed}\begin{verbatim}
  $ camllight my_little_caml
\end{verbatim}\end{boxed}

When the system returns its prompt, use the {\verb+#open+} directive to make
the library functions available:

\begin{boxed}\begin{verbatim}
  ##open "num";;
\end{verbatim}\end{boxed}

This library sets up a new type {\tt num} of arbitrary precision rational
numbers; we will just use the integer subset. CAML does not provide
overloading, so it is necessary to use different symbols for the usual
arithmetic operations over {\tt num}.

Note that small constants of type {\tt num} must be written as {\tt Int k}
rather than simply {\tt k}. In fact {\tt Int} is a type constructor for {\tt
num} used in the case when the number is a machine-representable integer.
Larger ones use {\tt Big\_int}.

The unary negation on type {\tt num} is written {\tt minus\_num}. Another handy
unary operator is {\tt abs\_num}, which finds absolute values, i.e. given $x$
returns $|x|$.

The usual binary operations are also available. The (truncating) division and
modulus function, called {\tt quo\_num} and {\tt mod\_num}, do not have infix
status. However most of the other binary operators are infixes, and the names
are derived from the usual ones by adding a slash `{\verb!/!}'. It is important
to realize that in general, one must use {\tt =/} to compare numbers for
equality. This is because the constructors of the type {\tt num} are, as
always, distinct by definition, yet in terms of the underlying meaning they may
overlap. For example we get different result from using truncating division and
true division on $2^{30}$ and $2$, even though they are numerically the same.
One uses type constructor {\tt Int} and the other {\tt Ratio}.

\begin{boxed}\begin{verbatim}
  #(Int 2 **/ Int 30) // Int 2 = quo_num (Int 2 **/ Int 30) (Int 2);;
  it : bool = false
  #(Int 2 **/ Int 30) // Int 2 =/ quo_num (Int 2 **/ Int 30) (Int 2);;
  it : bool = true
\end{verbatim}\end{boxed}

\noindent Here is a full list of the main infix binary operators:

\bigskip
\begin{tabular}{|l|l|l|}
\hline
Operator   & Type                             & Meaning               \\
\hline
{\tt **/}  & {\tt num -> num -> num}          & Exponentiation        \\
{\tt */}   & {\tt num -> num -> num}          & Multiplication        \\
{\tt +/}   & {\tt num -> num -> num}          & Addition              \\
{\tt -/}   & {\tt num -> num -> num}          & Subtraction           \\
{\tt =/}   & {\tt num -> num -> bool}         & Equality              \\
{\tt <>/}  & {\tt num -> num -> bool}         & Inequality            \\
{\tt </}   & {\tt num -> num -> bool}         & Less than             \\
{\tt <=/}  & {\tt num -> num -> bool}         & Less than or equal    \\
{\tt >/}   & {\tt num -> num -> bool}         & Greater than          \\
{\tt >=/}  & {\tt num -> num -> bool}         & Greater than or equal \\
\hline
\end{tabular}
\bigskip

\noindent Let us see some further examples:

\begin{boxed}\begin{verbatim}
  #Int 5 */ Int 14;;
  it : num = Int 70
  #Int 2 **/ Int 30;;
  it : num = Big_int <abstr>
  #(Int 2 **/ Int 30) // Int 2;;
  it : num = Ratio <abstr>
  #quo_num (Int 2 **/ Int 30) (Int 2);;
  it : num = Int 536870912
\end{verbatim}\end{boxed}

Note that large numbers are not printed. However we can always convert one to a
string using {\tt string\_of\_num}:

\begin{boxed}\begin{verbatim}
  #string_of_num(Int 2 ** Int 150);;
  - : string = "1427247692705959881058285969449495136382746624"
\end{verbatim}\end{boxed}

\noindent This also has an inverse, called naturally enough {\tt
num\_of\_string}.

\subsection{Basic operations}

Recall that our real numbers are supposed to be (represented by) functions
$\num \to \num$. In ML we will actually use {\tt int -> num}, since the inbuilt
type of integers is more than adequate for indexing the level of accuracy. Now
we can define some operations on reals. The most basic operation, which gets us
started, is to produce the real number corresponding to an integer. This is
easy:

\begin{boxed}\begin{verbatim}
  #let real_of_int k n = (Int 2 **/ Int n) */ Int k;;
  real_of_int : int -> int -> num = <fun>
  #real_of_int 23;;
  - : int -> num = <fun>
\end{verbatim}\end{boxed}

\noindent It is obvious that for any $k$ this obeys the desired approximation
criterion:

$$ |f_k(n) - 2^n k| = |2^n k - 2^n k| = 0 < 1 $$

\noindent Now we can define the first nontrivial operation, that of unary
negation:

\begin{boxed}\begin{lstlisting}
  let real_neg f n = minus_num(f n);;
\end{lstlisting}\end{boxed}

The compiler generalizes the type more than intended, but this will not trouble
us. It is almost as easy to see that the approximation criterion is preserved.
If we know that for each $n$:

$$ |f_x(n) - 2^n x| < 1 $$

\noindent then we have for any $n$:

\begin{eqnarray*}
|f_{-x}(n) - 2^n (-x)| & = & |-f_x(n) - 2^n (-x)|       \\
                       & = & |-(f_x(n) - 2^n x)|        \\
                       & = & |f_x(n) - 2^n x|           \\
                       & < & 1
\end{eqnarray*}

Similarly, we can define an `absolute value' function on real numbers, using
the corresponding function {\tt abs\_num} on numbers:

\begin{boxed}\begin{lstlisting}
  let real_abs f n = abs_num (f n);;
\end{lstlisting}\end{boxed}

The correctness of this is again straightforward, using the fact that $||x| -
|y|| \leq |x - y|$. Now consider the problem of adding two real numbers.
Suppose $x$ and $y$ are represented by $f_x$ and $f_y$ respectively. We could
define:

$$ f_{x + y}(n) = f_x(n) + f_y(n) $$

However this gives no guarantee that the approximation criterion is maintained;
we would have:

\begin{eqnarray*}
|f_{x + y}(n) - 2^n (x + y)| & = & |f_x(n) + f_y(n) - 2^n (x + y)|      \\
                             & \leq & |f_x(n) - 2^n x| + |f_y(n) - 2^n y|
\end{eqnarray*}

We can guarantee that the sum on the right is less than $2$, but not that it is
less than $1$ as required. Therefore, we need in this case to evaluate $x$ and
$y$ to {\em greater} accuracy than required in the answer. Suppose we define:

$$ f_{x + y}(n) = (f_x(n + 1) + f_y(n + 1)) / 2 $$

\noindent Now we have:

\begin{eqnarray*}
|f_{x + y}(n) - 2^n (x + y)|
& = & |(f_x(n + 1) + f_y(n + 1)) / 2 - 2^n (x + y)|             \\
& \leq & |f_x(n + 1) / 2 - 2^n x| + |f_y(n + 1) / 2 - 2^n y|    \\
& = & {1 \over 2} |f_x(n + 1) - 2^{n + 1} x| +
      {1 \over 2} |f_y(n + 1) - 2^{n + 1} y|                    \\
& < & {1 \over 2} 1 + {1 \over 2} 1 = 1
\end{eqnarray*}

Apparently this just gives the accuracy required. However we have implicitly
used real mathematical division above. Since the function is supposed to yield
an integer, we are obliged to round the quotient to an integer. If we just use
{\tt quo\_num}, the error from rounding this might be almost $1$, after which
we could never guarantee the bound we want, however accurately we evaluate the
arguments. However with a bit more care we can define a division function that
always returns the integer closest to the true result (or one of them in the
case of two equally close ones), so that the rounding error never exceeds $1
\over 2$. This could be done directly in integer arithmetic, but the most
straightforward coding is to use rational division followed by rounding to the
nearest integer, as there are built-in functions for both these operations:

\begin{boxed}\begin{verbatim}
  #let ndiv x y = round_num(x // y);;
  ndiv : num -> num -> num = <fun>
  ##infix "ndiv";;
  #(Int 23) ndiv (Int 5);;
  - : num = Int 5
  #(Int 22) ndiv (Int 5);;
  - : num = Int 4
  #(Int(-11)) ndiv (Int 4);;
  - : num = Int -3
  #(Int(-9)) ndiv (Int 4);;
  - : num = Int -2
\end{verbatim}\end{boxed}

\noindent Now if we define:

$$ f_{x + y}(n) = (f_x(n + 2) + f_y(n + 2)) \mbox{ ndiv } 4 $$

\noindent everything works:

\begin{eqnarray*}
|f_{x + y}(n) - 2^n (x + y)|
& =    & |((f_x(n + 2) + f_y(n + 2)) \mbox{ ndiv } 4) - 2^n (x + y)|    \\
& \leq & {1 \over 2} + |(f_x(n + 2) + f_y(n + 2)) / 4 - 2^n (x + y)|     \\
& =    & {1 \over 2} + {1 \over 4} |(f_x(n + 2) + f_y(n + 2)) -
                                     2^{n + 2}(x + y)|                  \\
& \leq & {1 \over 2} + {1 \over 4} |f_x(n + 2) - 2^{n + 2} x| +
                       {1 \over 4} |f_y(n + 2) - 2^{n + 2} y|           \\
& <    & {1 \over 2} + {1 \over 4} 1 + {1 \over 4} 1                    \\
&=     & 1
\end{eqnarray*}

\noindent Accordingly we make our definition:

\begin{boxed}\begin{lstlisting}
  let real_add f g n =
    (f(n + 2) +/ g(n + 2)) ndiv (Int 4);;
\end{lstlisting}\end{boxed}

We can define subtraction similarly, but the simplest approach is to build it
out of the functions that we already have:

\begin{boxed}\begin{verbatim}
  #let real_sub f g = real_add f (real_neg g);;
  real_sub : (num -> num) -> (num -> num) -> num -> num = <fun>
\end{verbatim}\end{boxed}

It is a bit trickier to define multiplication, inverses and division. However
the cases where we multiply or divide by an integer are much easier and quite
common. It is worthwhile implementing these separately as they can be made more
efficient. We define:

$$ f_{m x}(n) = (m f_x(n+p+1)) \mbox{ ndiv } 2^{p+1} $$

\noindent where $p$ is chosen so that $2^p \geq |m|$. For correctness, we have:

\begin{eqnarray*}
|f_{m x}(n) - 2^n (m x)|
& \leq & {1 \over 2} + |{m f_x(n+p+1) \over 2^{p+1}} - 2^n (m x)|       \\
& =    & {1 \over 2} + {|m| \over 2^{p+1}} |f_x(n+p+1) - 2^{n+p+1} x|   \\
& <    & {1 \over 2} + {|m| \over 2^{p+1}}                              \\
& \leq & {1 \over 2} + {1 \over 2} {|m| \over 2^p}                      \\
& \leq & {1 \over 2} + {1 \over 2} = 1
\end{eqnarray*}

In order to implement this, we need a function to find the appropriate $p$. The
following is crude but adequate:

\begin{boxed}\begin{lstlisting}
  let log2 =
    let rec log2 x y =
      if x </ Int 1 then y
      else log2 (quo_num x (Int 2)) (y + 1) in
    fun x -> log2 (x -/ Int 1) 0;;
\end{lstlisting}\end{boxed}

\noindent The implementation is simply:

\begin{boxed}\begin{lstlisting}
  let real_intmul m x n =
     let p = log2 (abs_num m) in
     let p1 = p + 1 in
     (m */ x(n + p1)) ndiv (Int 2 **/ Int p1);;
\end{lstlisting}\end{boxed}

\noindent For division by an integer, we define:

$$ f_{x / m}(n) = f_x(n) \mbox{ ndiv } m $$

For correctness, we can ignore the trivial cases when  $m = 0$, which should
never be used, and when $m = \pm 1$, since then the result is exact. Otherwise,
we assume $|f_x(n) - 2^n x| < 1$, so $|f_x(n) / m - 2^n x / m| < {1 \over |m|}
\leq {1 \over 2}$, which, together with the fact that $|f_x(n) \mbox{ ndiv } m
- f_x(n) / m| \leq {1 \over 2}$, yields the result. So we have:

\begin{boxed}\begin{lstlisting}
  let real_intdiv m x n =
     x(n) ndiv (Int m);;
\end{lstlisting}\end{boxed}

\subsection{General multiplication}

General multiplication is harder because the error in approximation for one
number is multiplied by the magnitude of the second number. Therefore, before
the final accuracies are calculated, a preliminary evaluation of the arguments
is required to determine their approximate magnitude. We proceed as follows.
Suppose that we want to evaluate $x + y$ to precision $n$. First we choose $r$
and $s$ so that $|r - s| \leq 1$ and $r + s = n + 2$. That is, both $r$ and $s$
are slightly more than half the required precision. We now evaluate $f_x(r)$
and $f_y(s)$, and select natural numbers $p$ and $q$ that are the corresponding
`binary logarithms', i.e. $|f_x(r)| \leq 2^p$ and $|f_y(s)| \leq 2^q$. If both
$p$ and $q$ are zero, then it is easy to see that we may return just $0$.
Otherwise, remember that either $p > 0$ or $q > 0$ as we will need this later.
Now set:

\begin{eqnarray*}
k &= &n + q - s + 3 =  q + r + 1           \\
l &= &n + p - r + 3 =  p + s + 1           \\
m &= &(k + l) - n =  p + q + 4
\end{eqnarray*}

We claim that $f_{x y}(n) = (f_x(k) f_y(l)) \mbox{ ndiv } 2^m$ has the right
error behaviour, i.e. $|f_{x y}(n) - 2^n (x y)| < 1$. If we write:

\begin{eqnarray*}
2^k x &= &f_x(k) + \delta        \\
2^l y &= &f_y(l) + \epsilon
\end{eqnarray*}

\noindent with $|\delta| < 1$ and $|\epsilon| < 1$, we have:

\begin{eqnarray*}
|f_{x y}(n) - 2^n (x y)|
 &\leq &{1 \over 2} + |{f_x(k) f_y(l) \over 2^m} -  2^n (x y)|                \\
 &= &{1 \over 2} + 2^{-m} |f_x(k) f_y(l) - 2^{k+l} x y|                       \\
 &= &{1 \over 2} + 2^{-m} |f_x(k) f_y(l) - (f_x(k) + \delta) (f_y(l) + \epsilon)|   \\
 &= &{1 \over 2} + 2^{-m} |\delta f_y(l) + \epsilon f_x(k) + \delta \epsilon| \\
 &\leq &{1 \over 2} + 2^{-m} (|\delta f_y(l)| + |\epsilon f_x(k)|
                                             + |\delta \epsilon|)         \\
 &\leq &{1 \over 2} + 2^{-m} (|f_y(l)| + |f_x(k)|  + |\delta \epsilon|)      \\
 &< &{1 \over 2} + 2^{-m} (|f_y(l)| + |f_x(k)| + 1)
\end{eqnarray*}

Now we have $|f_x(r)| \leq 2^p$, so $|2^r x| < 2^p + 1$. Thus $|2^k x| < 2^{q+1} (
2^p + 1)$, so $|f_x(k)| < 2^{q+1} (2^p + 1) + 1$, i.e. $|f_x(k)| \leq 2^{q+1} (2^p +
1)$. Similarly $|f_y(l)| \leq 2^{p+1} (2^q + 1)$. Consequently:

\begin{eqnarray*}
|f_y(l)| + |f_x(k)| + 1
&\leq &2^{p+1} (2^q + 1) + 2^{q+1} (2^p + 1) + 1                      \\
&= &2^{p+q+1} + 2^{p+1} + 2^{p+q+1} + 2^{q+1} + 1                     \\
&= &2^{p+q+2} + 2^{p+1} + 2^{q+1} + 1
\end{eqnarray*}

Now for our error bound we require $|f_y(l)| + |f_x(k)| + 1 \leq 2^{m-1}$, or
dividing by $2$ and using the discreteness of the integers:

$$ 2^{p+q+1} + 2^{p} + 2^{q} < 2^{p + q + 2} $$

We can write this as $(2^{p+q} + 2^p) + (2^{p+q} + 2^q) < 2^{p + q + 1} +
2^{p + q + 1}$, which is true because we have either $p > 0$ or $q > 0$. So at
last we are justified in defining:

\begin{boxed}\begin{lstlisting}
  let real_mul x y n =
    let n2 = n + 2 in
    let r = n2 / 2 in
    let s = n2 - r in
    let xr = x(r)
    and ys = y(s) in
    let p = log2 xr
    and q = log2 ys in
    if p = 0 & q = 0 then Int 0 else
    let k = q + r + 1
    and l = p + s + 1
    and m = p + q + 4 in
    (x(k) */ y(l)) ndiv (Int 2 **/ Int m);;
\end{lstlisting}\end{boxed}

\subsection{Multiplicative inverse}

Next we will define the multiplicative inverse function. In order to get any
sort of upper bound on this, let alone a good approximation, we need to get a
{\em lower} bound for the argument. In general, there is no better way than to
keep evaluating the argument with greater and greater accuracy until we can
bound it away from zero. We will use the following lemma to justify the
correctness of our procedure:

\begin{lemma}
If $2 e \geq n + k + 1$, $|f_x(k)| \geq 2^e$ and $|f_x(k) - 2^k x| < 1$, where
$f_x(k)$ is an integer and $e$, $n$ and $k$ are natural numbers, then if we
define

$$ f_y(n) = 2^{n+k} \mbox{ ndiv } f_x(k) $$

\noindent we have $|f_y(n) - 2^n x^{-1}| < 1$, i.e. the required bound.

\proof The proof is rather tedious and will not be given in full. We just
sketch the necessary case splits. If $|f_x(k)| > 2^e$ then a straightforward
analysis gives the result; the rounding in {\tt ndiv} gives an error of at most
$1 \over 2$, and the remaining error is $< {1 \over 2}$. If $|f_x(k)| = 2^e$ but
$n + k \geq e$, then although the second component of the error may now be
twice as much, i.e. $< 1$, there is no rounding error because $f_x(k) = \pm 2^e$
divides into $2^{n+k}$ exactly. (We use here the fact that $2^e - 1 \leq
2^{e-1}$, because since $2 e \geq n + k + 1$, $e$ cannot be zero.) Finally, if
$|f_x(k)| = 2^e$ and $n + k < e$, we have $|f_y(n) - 2^n {1 \over x}| < 1$ because
both $|f_y(n)| \leq 1$ and $0 < |2^n {1 \over x}| < 1$, and both these numbers
have the same sign. \qed

\end{lemma}

Now suppose we wish to find the inverse of $x$ to accuracy $n$. First we
evaluate $f_x(0)$. There are two cases to distinguish:

\begin{enumerate}

\item If $|f_x(0)| > 2^r$ for some natural number $r$, then choose the least
natural number $k$ (which may well be zero) such that $2 r + k \geq n + 1$, and
set $e = r + k$. We now return $2^{n+k} \mbox{ ndiv } f_x(k)$. It is easy to
see that the conditions of the lemma are satisfied. Since $|f_x(0)| \geq 2^r +
1$ we have $|x| > 2^r$, and so $|2^k x| > 2^{r + k}$. This means $|f_x(k)| >
2^{r+k} - 1$, and as $f_x(k)$ is an integer, $|f_x(k)| \geq 2^{r + k} = 2^e$ as
required. The condition that $2 e \geq n = k + 1$ is easy to check. Note that
if $r \geq n$ we can immediately deduce that $f_y(n) = 0$ is a valid
approximation.

\item If $|f_x(0)| \leq 1$, then we call the function `{\tt msd}' that returns
the least $p$ such that $|f_x(p)| > 1$. Note that this may in general fail to
terminate if $x = 0$. Now we set $e = n + p + 1$ and $k = e + p$, and return
$2^{n+k} \mbox{ ndiv } f_x(k)$. Once again the conditions for the lemma are
satisfied. Since $|f_x(p)| \geq 2$, we have $|2^p x| > 1$, i.e. $|x| > {1 \over
{2^p}}$. Hence $|2^k x| > 2^{k-p} = 2^e$, and so $|f_x(k)| > 2^e - 1$, i.e.
$|f_x(k)| \geq 2^e$.

\end{enumerate}

\noindent To implement this, we first define the {\tt msd} function:

\begin{boxed}\begin{lstlisting}
  let msd =
    let rec msd n x =
      if abs_num(x(n)) >/ Int 1 then n else msd (n + 1) x in
    msd 0;;
\end{lstlisting}\end{boxed}

\noindent and then translate the above mathematics into a simple program:

\begin{boxed}\begin{lstlisting}
  let real_inv x n =
    let x0 = x(0) in
    let k =
      if x0 >/ Int 1 then
        let r = log2 x0 - 1 in
        let k0 = n + 1 - 2 * r in
        if k0 < 0 then 0 else k0
      else
        let p = msd x in
        n + 2 * p + 1 in
    (Int 2 **/ Int (n + k)) ndiv (x(k));;
\end{lstlisting}\end{boxed}

\noindent Now, of course, it is straightforward to define division:

\begin{boxed}\begin{lstlisting}
  let real_div x y = real_mul x (real_inv y);;
\end{lstlisting}\end{boxed}

\subsection{Ordering relations}

The interesting things about all these functions is that they are in fact
uncomputable in general. The essential point is that it is impossible to decide
whether a given number is zero. We can keep approximating it to greater and
greater accuracy, but if the approximations always return $0$, we cannot tell
whether it is going to do so indefinitely or whether the next step will give a
nonzero answer.\footnote{For a proof that it is uncomputable, simply wrap up
the halting problem, by defining $f(n) = 1$ if a certain Turing machine has
halted after no more than $n$ iterations, and $f(n) = 0$ otherwise.} If $x$ is
not zero, then the search will eventually terminate, but if $x$ is zero, it
will run forever.

Accepting this, it is not difficult to define the relational operators. To
decide the ordering relation of $x$ and $y$ it suffices to find an $n$ such
that $|x_n - y_n| \geq 2$. For example, if $x_n \geq y_n + 2$ we have

$$ 2^n x > x_n - 1 \geq y_n + 1 > 2^n y $$

\noindent and so $x > y$. We first write a general procedure to perform this
search, and then define all the orderings in terms of it. Note that the only
way to arrive at the reflexive orderings is to justify the irreflexive version!

\begin{boxed}\begin{lstlisting}
  let separate = 
    let rec separate n x y =
      let d = x(n) -/ y(n) in
      if abs_num(d) >/ Int 1 then d
      else separate (n + 1) x y in
  separate 0;;

  let real_gt x y = separate x y >/ Int 0;;
  let real_ge x y = real_gt x y;;
  let real_lt x y = separate x y </ Int 0;;
  let real_le x y = real_lt x y;;
\end{lstlisting}\end{boxed}

\subsection{Caching}

In order to try out all the above functions, we would like to be able to print
out a decimal representation of some approximation to a real. This is not hard,
using some more standard facilities in the CAML library. If $d$ decimal places
are desired, then we make $n$ such that $2^n > 10^d$ and hence the accuracy
corresponds at least to the number of digits printed.

\begin{boxed}\begin{lstlisting}
  let view x d =
    let n = 4 * d in
    let out = x(n) // (Int 2 **/ Int n) in
    approx_num_fix d out;;
\end{lstlisting}\end{boxed}

Now we can test out some simple examples, which seem to work:

\begin{boxed}\begin{verbatim}
  #let x = real_of_int 3;;
  x : int -> num = <fun>
  #let xi = real_inv x;;
  xi : int -> num = <fun>
  #let wun = real_mul x xi;;
  wun : int -> num = <fun>
  #view x 20;;
  it : string = "3.00000000000000000000"
  #view xi 20;;
  it : string = ".33333333333333333333"
  #view wun 20;;
  it : string = "1.00000000000000000000"
\end{verbatim}\end{boxed}

However there is a subtle and serious bug in our implementation, which shows
itself if we try larger expressions. The problem is that subexpressions can be
evaluated many times. Most obviously, this happens if they occur several times
in the input expression. But even if they don't, the multiplication, and even
more so, the inverse, require trial evaluations at differing accuracies. As the
evaluation propagates down to the bottom level, there is often an exponential
buildup of reevaluations. For example, the following is slow --- it takes
several seconds.

\begin{boxed}\begin{verbatim}
  #let x1 = real_of_int 1 in
   let x2 = real_mul x1 x1 in
   let x3 = real_mul x2 x2 in
   let x4 = real_mul x3 x3 in
   let x5 = real_mul x4 x4 in
   let x6 = real_mul x5 x5 in
   let x7 = real_mul x6 x6 in
   view x7 10;;
   - : string = "+1.0000000000"
\end{verbatim}\end{boxed}

We can fix this problem using the idea of {\em caching} or {\em memo
functions} \cite{michie-memo}. We give each function a reference cell to
remember the most accurate version already calculated. If there is a second
request for the same accuracy, it can be returned immediately without further
evaluation. What's more, we can always construct a lower-level approximation,
say $n$, from a higher one, say $n + k$ with $k \geq 1$. If we know that
$|f_x(n + k) - 2^{n + k} x| < 1$ then we have:

\begin{eqnarray*}
|f_x(n + k) \mbox{ ndiv } 2^ k - 2^n x|
& \leq & {1 \over 2} + |{f_x(n + k) \over 2^k} - 2^n x|                 \\
& =    & {1 \over 2} + {1 \over 2^k} |f_x(n + k) - 2^{n + k} x|         \\
& <    & {1 \over 2} + {1 \over 2^k}                                    \\
& \leq & 1
\end{eqnarray*}

Hence we are always safe in returning $f_x(n + k) \mbox{ ndiv } 2^k$. We can
implement this memo technique via a generic function {\tt memo} to be inserted
in each of the previous functions:

\begin{boxed}\begin{lstlisting}
  let real_of_int k = memo (fun n -> (Int 2 **/ Int n) */ Int k);;

  let real_neg f = memo (fun n -> minus_num(f n));;

  let real_abs f = memo (fun n -> abs_num (f n));;

  let real_add f g = memo (fun n ->
     (f(n + 2) +/ g(n + 2)) ndiv (Int 4));;

  let real_sub f g = real_add f (real_neg g);;

  let real_intmul m x = memo (fun n ->
     let p = log2 (abs_num m) in
     let p1 = p + 1 in
     (m */ x(n + p1)) ndiv (Int 2 **/ Int p1));;

  let real_intdiv m x = memo (fun n ->
     x(n) ndiv (Int m));;

  let real_mul x y = memo (fun n ->
    let n2 = n + 2 in
    let r = n2 / 2 in
    let s = n2 - r in
    let xr = x(r)
    and ys = y(s) in
    let p = log2 xr
    and q = log2 ys in
    if p = 0 & q = 0 then Int 0 else
    let k = q + r + 1
    and l = p + s + 1
    and m = p + q + 4 in
    (x(k) */ y(l)) ndiv (Int 2 **/ Int m));;

  let real_inv x = memo (fun n ->
    let x0 = x(0) in
    let k =
      if x0 >/ Int 1 then
        let r = log2 x0 - 1 in
        let k0 = n + 1 - 2 * r in
        if k0 < 0 then 0 else k0
      else
        let p = msd x in
        n + 2 * p + 1 in
    (Int 2 **/ Int (n + k)) ndiv (x(k)));;

  let real_div x y = real_mul x (real_inv y);;
\end{lstlisting}\end{boxed}

\noindent where

\begin{boxed}\begin{lstlisting}
  let memo f =
    let mem = ref (-1,Int 0) in
    fun n -> let (m,res) = !mem in
             if n <= m then
               if m = n then res
               else res ndiv (Int 2 **/ Int(m - n))
             else
               let res = f n in
               mem := (n,res); res;;
\end{lstlisting}\end{boxed}

Now the above sequence of multiplications is instantaneous. Here are a few more
examples:

\begin{boxed}\begin{verbatim}
  #let pi1 = real_div (real_of_int 22) (real_of_int 7);;
  pi1 : int -> num = <fun>
  #view pi1 10;;
  it : string = "+3.1428571429"
  #let pi2 = real_div (real_of_int 355) (real_of_int 113);;
  pi2 : int -> num = <fun>
  #view pi2 10;;
  it : string = "+3.1415929204"
  #let pidiff = real_sub pi1 pi2;;
  pidiff : int -> num = <fun>
  #view pidiff 20;;
  it : string = "+0.00126422250316055626"
  #let ipidiff = real_inv pidiff;;
  ipidiff : int -> num = <fun>
  #view ipidiff 20;;
  it : string = "+791.00000000000000000000"
\end{verbatim}\end{boxed}

Of course, everything we have done to date could be done with rational
arithmetic. Actually, it may happen that our approach is more efficient in
certain situations since we avoid calculating numbers that might have immense
numerators and denominators when we just need an approximation. Nevertheless,
the present approach really comes into its own when we want to define
transcendental functions like $exp$, $sin$ etc. We will not cover this in
detail for lack of space, but it makes an interesting exercise. One approach is
to use truncated Taylor series. Note that finite summations can be evaluated
directly rather than by iterating the addition function; this leads to much
better error behaviour.

\section{Prolog and theorem proving}

The language Prolog is popular in Artificial Intelligence research, and is used
in various practical applications such as expert systems and intelligent
databases. Here we will show how the main mechanism of Prolog, namely
depth-first search through a database of rules using unification and
backtracking, can be implemented in ML. We do not pretend that this is a
full-blown implementation of Prolog, but it gives an accurate picture of the
general flavour of the language, and we will be able to run some simple
examples in our system.

\subsection{Prolog terms}

Prolog code and data is represented using a uniform system of first order
terms. We have already defined a type of terms for our mathematical
expressions, and associated parsers and printers. Here we will use something
similar, but not quite the same. First of all, it simplifies some of the code
if we treat constants as nullary functions, i.e. functions that take an empty
list of arguments. Accordingly we define:

\begin{boxed}\begin{lstlisting}
  type term = Var of string
            | Fn of string * (term list);;
\end{lstlisting}\end{boxed}

Where we would formerly have used {\verb+Const s+}, we will now use
{\verb+Fn(s,[])+}. Note that we will treat functions of different arities
(different numbers of arguments) as distinct, even if they have the same name.
Thus there is no danger of our confusing constants with true functions.

\subsection{Lexical analysis}

In order to follow Prolog conventions, which include case sensitivity, we also
modify lexical analysis. We will not attempt to conform exactly to the full
details of Prolog, but one feature is very important: alphanumeric identifiers
that begin with an {\em upper case} letter or an underscore are treated as
variables, while other alphanumeric identifiers, along with numerals, are
treated as constants. For example {\verb+X+} and {\verb+Answer+} are variables
while {\verb+x+} and {\verb+john+} are constants. We will lump all symbolic
identifiers together as constants too, but we will distinguish the punctuation
symbols: left and right brackets, commas and semicolons. Non-punctuation
symbols are collected together into the longest strings possible, so symbolic
identifiers need not consist only of one character.

\begin{boxed}\begin{lstlisting}
  type token = Variable of string
             | Constant of string
             | Punct of string;;
\end{lstlisting}\end{boxed}

\noindent The lexer therefore looks like this:

\begin{boxed}\begin{lstlisting}
  let lex =
    let several p = many (some p) in
    let collect(h,t) = h^(itlist (prefix ^) t "") in
    let upper_alpha s = "A" <= s & s <= "Z" or s = "_"
    and lower_alpha s = "a" <= s & s <= "z" or "0" <= s & s <= "9"
    and punct s = s = "(" or s = ")" or s = "[" or s = "]"
                  or s = "," or s = "."
    and space s = s = " " or s = "\n" or s = "\t" in
    let alphanumeric s = upper_alpha s or lower_alpha s in
    let symbolic s = not space s & not alphanumeric s & not punct s in
    let rawvariable =
      some upper_alpha ++ several alphanumeric >> (Variable o collect)
    and rawconstant =
      (some lower_alpha ++ several alphanumeric ||
       some symbolic ++ several symbolic) >> (Constant o collect)
    and rawpunct = some punct >>  Punct in
    let token =
      (rawvariable || rawconstant || rawpunct) ++
      several space >> fst in
    let tokens = (several space ++ many token) >> snd in
    let alltokens = (tokens ++ finished) >> fst in
    fst o alltokens o explode;;
\end{lstlisting}\end{boxed}

\noindent For example:

\begin{boxed}\begin{verbatim}
  #lex "add(X,Y,Z) :- X is Y+Z.";;
  - : token list =
   [Constant "add"; Punct "("; Variable "X"; Punct ",";
    Variable "Y"; Punct ","; Variable "Z"; Punct ")";
    Constant ":-"; Variable "X"; Constant "is"; Variable "Y";
    Constant "+"; Variable "Z"; Punct "."]
\end{verbatim}\end{boxed}

\subsection{Parsing}

The basic parser is pretty much the same as before; the printer is exactly the
same. The main modification to the printer is that we allow Prolog lists to be
written in a more convenient notation. Prolog has `.' and `nil' corresponding
to ML's `{\tt ::}' and `{\tt []}', and we set up the parser so that lists can
be written much as in ML, e.g. `{\tt [1,2,3]}'. We also allow the Prolog
notation `{\verb+[H|T]+}' instead of `{\verb+cons(H,T)+}', typically used for
pattern-matching. After the basic functions:

\begin{boxed}\begin{lstlisting}
  let variable =
    fun (Variable s::rest) -> s,rest
      | _ -> raise Noparse;;

  let constant =
    fun (Constant s::rest) -> s,rest
      | _ -> raise Noparse;;
\end{lstlisting}\end{boxed}

\noindent we have a parser for terms and also for Prolog rules, of these forms:

\begin{eqnarray*}
term\mbox{.}    & &                                             \\
term            & \mbox{ :- } & term,\ldots,term\mbox{.}
\end{eqnarray*}

\noindent The parsers are as follows:

\begin{boxed}\begin{lstlisting}
  let rec atom input
         = (constant ++ a (Punct "(") ++ termlist ++ a (Punct ")")
                >> (fun (((name,_),args),_) -> Fn(name,args))
         || constant
                >> (fun s -> Fn(s,[]))
         || variable
                >> (fun s -> Var s)
         || a (Punct "(") ++ term ++ a (Punct ")")
                >> (snd o fst)
         || a (Punct "[") ++ list
                >> snd) input
    and term input = precedence (!infixes) atom input
    and termlist input
         = (term ++ a (Punct ",") ++ termlist
                >> (fun ((h,_),t) -> h::t)
         || term
                >> (fun h -> [h])) input
    and list input
         = (term ++ (a (Constant "|") ++ term ++ a (Punct "]")
                          >> (snd o fst)
                  || a (Punct ",") ++ list
                           >> snd
                  || a (Punct "]")
                          >> (K (Fn("[]",[]))))
                >> (fun (h,t) -> Fn(".",[h; t]))
         || a (Punct "]")
                >> (K (Fn("[]",[])))) input
    and rule input
        = (term ++ (a (Punct ".")
                          >> (K [])
                 || a (Constant ":-") ++ term ++
                    many (a (Punct ",") ++ term >> snd) ++
                    a (Punct ".")
                          >> (fun (((_,h),t),_) -> h::t))) input;;

  let parse_term = fst o (term ++ finished >> fst) o lex;;

  let parse_rules = fst o (many rule ++ finished >> fst) o lex;;
\end{lstlisting}\end{boxed}

\subsection{Unification}

Prolog uses a set of rules to solve a current {\em goal} by trying to match one
of the rules against the goal. A rule consisting of a single term can solve a
goal immediately. In the case of a rule $term \mbox{ :- }
term_1,\ldots,term_n\mbox{.}$, if the goal matches $term$, then Prolog needs to
solve each $term_i$ as a subgoal in order to finish the original goal.

However, goals and rules do not have to be exactly the same. Instead, Prolog
assigns variables in both the goal and the rule to make them match up, a
process known as {\em unification}. This means that we can end up proving a
special case of the original goal, e.g. $P(f(X))$ instead of
$P(Y)$. For example:

\begin{itemize}

\item To unify $f(g(X),Y)$ and $f(g(a),X)$ we can set
$X = a$ and $Y = a$. Then both terms are $f(g(a),a)$.

\item To unify $f(a,X,Y)$ and $f(X,a,Z)$ we can set $X
= a$ and $Y = Z$, and then both terms are $f(a,a,Z)$.

\item It is impossible to unify $f(X)$ and $X$.

\end{itemize}

In general, unifiers are not unique. For example, in the second example we
could choose to set $Y = f(b)$ and $Z = f(b)$. However one can always choose
one that is {\em most general}, i.e. any other unification can be reached from
it by further instantiation (compare most general types in ML). In order to
find it, roughly speaking, one need only descend the two terms recursively
in parallel, and on finding a variable on either side, assigns it to whatever
the term on the other side is. One also needs to check that the variable hasn't
already been assigned to something else, and that it doesn't occur in the term
being assigned to it (as in the last example above). A simple implementation of
this idea follows. We maintain an association list giving instantiations
already made, and we look each variable up to see if it is already assigned
before proceeding. We use the existing instantiations as an accumulator.

\begin{boxed}\begin{lstlisting}
  let rec unify tm1 tm2 insts =
    match tm1 with
      Var(x) ->
        (try let tm1' = assoc x insts in
             unify tm1' tm2 insts
         with Not_found ->
             augment (x,tm2) insts)
    | Fn(f1,args1) ->
        match tm2 with
          Var(y) ->
            (try let tm2' = assoc y insts in
                 unify tm1 tm2' insts
             with Not_found ->
                 augment (y,tm1) insts)
        | Fn(f2,args2) ->
            if f1 = f2
            then itlist2 unify args1 args2 insts
            else raise (error "functions do not match");;
\end{lstlisting}\end{boxed}

\noindent where the instantiation lists need to be augmented with some care to
avoid the so-called `occurs check' problem. We must disallow instantiation of
$X$ to a nontrivial term involving $X$, as in the third example above. Most
real Prologs ignore this, either for (claimed) efficiency reasons or in order
to allow weird cyclic datastructures instead of simple first order terms.

\begin{boxed}\begin{lstlisting}
  let rec occurs_in x =
    fun (Var y) -> x = y
      | (Fn(_,args)) -> exists (occurs_in x) args;;

  let rec subst insts = fun
     (Var y) -> (try assoc y insts with Not_found -> tm)
   | (Fn(f,args)) -> Fn(f,map (subst insts) args);;

  let raw_augment =
    let augment1 theta (x,s) =
      let s' = subst theta s in
      if occurs_in x s & not(s = Var(x))
      then raise (error "Occurs check")
      else (x,s') in
    fun p insts -> p::(map (augment1 [p]) insts);;

  let augment (v,t) insts =
    let t' = subst insts t in match t' with
      Var(w) -> if w <= v then
                  if w = v then insts
                  else raw_augment (v,t') insts
                else raw_augment (w,Var(v)) insts
    | _ -> if occurs_in v t'
           then raise (error "Occurs check")
           else raw_augment (v,t') insts;;
\end{lstlisting}\end{boxed}

\subsection{Backtracking}

Prolog proceeds by depth-first search, but it may backtrack: even if a rule
unifies successfully, if all the remaining goals cannot be solved under the
resulting instantiations, then another initial rule is tried. Thus we consider
the whole list of goals rather than one at a time, to give the right control
strategy.

\begin{boxed}\begin{lstlisting}
  let rec first f =
    fun [] -> raise (error "No rules applicable")
      | (h::t) -> try f h with error _ -> first f t;;

  let rec expand n rules insts goals =
    first (fun rule ->
      if goals = [] then insts else
      let conc,asms =
        rename_rule (string_of_int n) rule in
      let insts' = unify conc (hd goals) insts in
      let local,global = partition
        (fun (v,_) -> occurs_in v conc or
                exists (occurs_in v) asms) insts' in
      let goals' = (map (subst local) asms) @
                   (tl goals) in
      expand (n + 1) rules global goals') rules;;
\end{lstlisting}\end{boxed}

Here we use a function `rename' to generate fresh variables for the rules each
time:

\begin{boxed}\begin{verbatim}
  let rec rename s =
    fun (Var v) -> Var("~"^v^s)
      | (Fn(f,args)) -> Fn(f,map (rename s) args);;

  let rename_rule s (conc,asms) =
    (rename s conc,map (rename s) asms);;
\end{verbatim}\end{boxed}

Finally, we can package everything together in a function {\tt prolog} that
tries to solve a given goal using the given rules:

\begin{boxed}\begin{verbatim}
  type outcome = No | Yes of (string * term) list;;

  let prolog rules goal =
    try let insts = expand 0 rules [] [goal] in
        Yes(filter (fun (v,_) -> occurs_in v goal)
                   insts)
    with error _ -> No;;
\end{verbatim}\end{boxed}

This says either that the goal cannot be solved, or that it can be solved with
the given instantiations. Note that we only return one answer in this latter
case, but this is easy to change if desired.

\subsection{Examples}

We can use the Prolog interpreter just written to try some simple examples from
Prolog textbooks. For example:

\begin{boxed}\begin{verbatim}
  #let rules = parse_rules
    "male(albert).
     male(edward).
     female(alice).
     female(victoria).
     parents(edward,victoria,albert).
     parents(alice,victoria,albert).
     sister_of(X,Y) :-
       female(X),
       parents(X,M,F),
       parents(Y,M,F).";;
  rules : (term * term list) list =
   [`male(albert)`, []; `male(edward)`, [];
    `female(alice)`, []; `female(victoria)`, [];
    `parents(edward,victoria,albert)`, [];
    `parents(alice,victoria,albert)`, [];
    `sister_of(X,Y)`,
      [`female(X)`; `parents(X,M,F)`; `parents(Y,M,F)`]]
  #prolog rules ("sister_of(alice,edward)");;
  - : outcome = Yes []
  #prolog rules (parse_term "sister_of(alice,X)");;
  - : outcome = Yes ["X", `edward`]
  #prolog rules (parse_term "sister_of(X,Y)");;
  - : outcome = Yes ["Y", `edward`; "X", `alice`]
\end{verbatim}\end{boxed}

The following are similar to some elementary ML list operations. Since Prolog
is relational rather than functional, it is possible to use Prolog queries in a
more flexible way, e.g. ask what arguments would give a certain result, rather
than vice versa:

\begin{boxed}\begin{verbatim}
  #let r = parse_rules
    "append([],L,L).
     append([H|T],L,[H|A]) :- append(T,L,A).";;
  r : (term * term list) list =
   [`append([],L,L)`, [];
    `append(H . T,L,H . A)`, [`append(T,L,A)`]]
  #prolog r (parse_term "append([1,2],[3],[1,2,3])");;
  - : outcome = Yes []
  #prolog r (parse_term "append([1,2],[3,4],X)");;
  - : outcome = Yes ["X", `1 . (2 . (3 . (4 . [])))`]
  #prolog r (parse_term "append([3,4],X,X)");;
  - : outcome = No
  #prolog r (parse_term "append([1,2],X,Y)");;
  - : outcome = Yes ["Y", `1 . (2 . X)`]
\end{verbatim}\end{boxed}

In such cases, Prolog seems to be showing an impressive degree of intelligence.
However under the surface it is just using a simple search strategy, and this
can easily be thwarted. For example, the following loops indefinitely:

\begin{boxed}\begin{verbatim}
  #prolog r (parse_term "append(X,[3,4],X)");;
\end{verbatim}\end{boxed}

\subsection{Theorem proving}

Prolog is acting as a simple theorem prover, using a database of logical facts
(the rules) in order to prove a goal. However it is rather limited in the facts
it can prove, partly because its depth-first search strategy is incomplete, and
partly because it can only make logical deductions in certain patterns. It is
possible to make Prolog-like systems that are more powerful, e.g. see
\citeN{stickel-pttp}. In what follows, we will just show how to build a more
capable theorem prover using essentially similar technology, including the
unification code and an identical backtracking strategy.

In particular, unification is an effective way of deciding how to specialize
universally quantified variables. For example, given the facts that
$\all{X} p(X) \Imp q(X)$ and $p(f(a))$, we can unify the two
expressions involving $p$ and thus discover that we need to set
$X$ to $f(a)$. By contrast, the very earliest theorem provers tried all
possible terms built up from the available constants and functions (the
`Herbrand base').

Usually, depth-first search would go into an infinite loop, so we need to
modify the Prolog strategy slightly. We will use {\em depth first iterative
deepening}. This means that the search depth has a hard limit, and attempts to
exceed it cause backtracking. However, if no proof is found at a given depth,
the bound is increased and another attempt is made. Thus, first one searches
for proofs of depth $1$, and if that fails, searches for one of depth $2$, then
depth $3$ and so on. For `depth' one can use various parameters, e.g. the
height or overall size of the search tree; we will use the number of unifiable
variables introduced.

\subsubsection*{Manipulating formulas}

We will simply use our standard first order terms to denote formulas,
introducing new constants for the logical operators. Many of these are written
infix.

\begin{center}
\begin{tabular}{|l|l|}
\hline
Operator   & Meaning                             \\
\hline
{\verb+~+}(p)    & not p                         \\
p {\verb+&+} q   & p and q                       \\
p {\verb+|+} q   & p or q                        \\
p {\verb+-->+} q & p implies q                   \\
p {\verb+<->+} q & p if and only if q            \\
forall(X,p)      & for all X, p                  \\
exists(X,p)      & there exists an X such that p \\
\hline
\end{tabular}
\end{center}

An alternative would be to introduce a separate type of formulas, but this
would require separate parsing and printing support. We will avoid this, for
the sake of simplicity.

\subsubsection*{Preprocessing formulas}

It's convenient if the main part of the prover need not cope with implications
and `if and only if's. Therefore we first define a function that eliminates
these in favour of the other connectives.

\begin{boxed}\begin{lstlisting}
  let rec proc tm =
    match tm with
      Fn("~",[t]) -> Fn("~",[proc t])
    | Fn("&",[t1; t2]) -> Fn("&",[proc t1; proc t2])
    | Fn("|",[t1; t2]) -> Fn("|",[proc t1; proc t2])
    | Fn("-->",[t1; t2]) ->
          proc (Fn("|",[Fn("~",[t1]); t2]))
    | Fn("<->",[t1; t2]) ->
          proc (Fn("&",[Fn("-->",[t1; t2]);
                        Fn("-->",[t2; t1])]))
    | Fn("forall",[x; t]) -> Fn("forall",[x; proc t])
    | Fn("exists",[x; t]) -> Fn("exists",[x; proc t])
    | t -> t;;
\end{lstlisting}\end{boxed}

The next step is to push the negations down the formula, putting it into
so-called `negation normal form' (NNF). We define two mutually recursive
functions that create NNF for a formula, and
its negation.

\begin{boxed}\begin{lstlisting}
  let rec nnf_p tm =
    match tm with
      Fn("~",[t]) -> nnf_n t
    | Fn("&",[t1; t2]) -> Fn("&",[nnf_p t1; nnf_p t2])
    | Fn("|",[t1; t2]) -> Fn("|",[nnf_p t1; nnf_p t2])
    | Fn("forall",[x; t]) -> Fn("forall",[x; nnf_p t])
    | Fn("exists",[x; t]) -> Fn("exists",[x; nnf_p t])
    | t -> t

  and nnf_n tm =
    match tm with
      Fn("~",[t]) -> nnf_p t
    | Fn("&",[t1; t2]) -> Fn("|",[nnf_n t1; nnf_n t2])
    | Fn("|",[t1; t2]) -> Fn("&",[nnf_n t1; nnf_n t2])
    | Fn("forall",[x; t]) -> Fn("exists",[x; nnf_n t])
    | Fn("exists",[x; t]) -> Fn("forall",[x; nnf_n t])
    | t -> Fn("~",[t]);;
\end{lstlisting}\end{boxed}

We will convert the negation of the input formula into negation normal form,
and the main prover will then try to derive a contradiction from it. This will
suffice to prove the original formula.

\subsubsection*{The main prover}

At each stage, the prover has a current formula, a list of formulas yet to be
considered, and a list of literals. It is trying to reach a contradiction. The
following strategy is employed:

\begin{itemize}

\item If the current formula is `p {\verb+&+} q', then consider
`p' and `q' separately, i.e. make `p' the current formula and add
`q' to the formulas to be considered.

\item If the current formula is `p {\verb+|+} q', then try to get a
contradiction from `p' and then one from `q'.

\item If the current formula is `forall(X,p)', invent a new variable to
replace `X': the right value can be discovered later by unification.

\item If the current formula is `exists(X,p)', invent a new constant to
replace `X'.

\item Otherwise, the formula must be a literal, so try to unify it with a
contradictory literal.

\item If this fails, add it to the list of literals, and proceed with the next
formula.

\end{itemize}

We desire a similar backtracking strategy to Prolog: only if the current
instantiation allow all remaining goals to be solved do we accept it. We could
use lists again, but instead we use {\em continuations}. A continuation is a
function that is passed to another function and can be called from within it to
`perform the rest of the computation'. In our case, it takes a list of
instantiations and tries to solve the remaining goals under these
instantiations. Thus, rather than explicitly trying to solve all remaining
goals, we simply try calling the continuation.

\begin{boxed}\begin{lstlisting}
  let rec prove fm unexp pl nl n cont i =
    if n < 0 then raise (error "No proof") else
    match fm with
      Fn("&",[p;q]) ->
          prove p (q::unexp) pl nl n cont i
    | Fn("|",[p;q]) ->
          prove p unexp pl nl n
          (prove q unexp pl nl n cont) i
    | Fn("forall",[Var x; p]) ->
          let v = mkvar() in
          prove (subst [x,Var v] p) (unexp@[fm]) pl nl (n - 1) cont i
    | Fn("exists",[Var x; p]) ->
          let v = mkvar() in
          prove (subst [x,Fn(v,[])] p) unexp pl nl (n - 1) cont i
    | Fn("~",[t]) ->
          (try first (fun t' -> let i' = unify t t' i in
                                cont i') pl
           with error _ ->
              prove (hd unexp) (tl unexp) pl (t::nl) n cont i)
    | t ->
          (try first (fun t' -> let i' = unify t t' i in
                                cont i') nl
           with error _ ->
              prove (hd unexp) (tl unexp) (t::pl) nl n cont i);;
\end{lstlisting}\end{boxed}

\noindent We set up the final prover as follows:

\begin{boxed}\begin{lstlisting}
  let prover =
    let rec prove_iter n t =
      try let insts = prove t [] [] [] n I [] in
          let globinsts = filter
            (fun (v,_) -> occurs_in v t) insts in
          n,globinsts
      with error _ -> prove_iter (n + 1) t in
    fun t -> prove_iter 0 (nnf_n(proc(parse_term t)));;
\end{lstlisting}\end{boxed}

This implements the iterative deepening strategy. It tries to find the proof
with the fewest universal instantiations; if it succeeds, it returns the number
required and any toplevel instantiations, throwing away instantiations for
internally created variables.

\subsubsection*{Examples}

Here are some simple examples from \citeN{pelletier-problems}.

\begin{boxed}\begin{verbatim}
  #let P1 = prover "p --> q <-> ~(q) --> ~(p)";;
  P1 : int * (string * term) list = 0, []
  #let P13 = prover "p | q & r <-> (p | q) & (p | r)";;
  P13 : int * (string * term) list = 0, []
  #let P16 = prover "(p --> q) | (q --> p)";;
  P16 : int * (string * term) list = 0, []
  #let P18 = prover "exists(Y,forall(X,p(Y)-->p(X)))";;
  P18 : int * (string * term) list = 2, []
  #let P19 = prover "exists(X,forall(Y,forall(Z,
                  (p(Y)-->q(Z))-->p(X)-->q(X))))";;
  P19 : int * (string * term) list = 6, []
\end{verbatim}\end{boxed}

\noindent A bigger example is the following:

\begin{boxed}\begin{verbatim}
  #let P55 = prover
    "lives(agatha) & lives(butler) & lives(charles) &
     (killed(agatha,agatha) | killed(butler,agatha) |
      killed(charles,agatha)) &
     (forall(X,forall(Y,
       killed(X,Y) --> hates(X,Y) & ~(richer(X,Y))))) &
     (forall(X,hates(agatha,X)
               --> ~(hates(charles,X)))) &
     (hates(agatha,agatha) & hates(agatha,charles)) &
     (forall(X,lives(X) & ~(richer(X,agatha))
               --> hates(butler,X))) &
     (forall(X,hates(agatha,X) --> hates(butler,X))) &
     (forall(X,~(hates(X,agatha)) | ~(hates(X,butler))
               | ~(hates(X,charles))))
     --> killed(agatha,agatha)";;
  P55 : int * (string * term) list = 6, []
\end{verbatim}\end{boxed}

\noindent In fact, the prover can work out `whodunit':

\begin{boxed}\begin{verbatim}
  #let P55' = prover
    "lives(agatha) & lives(butler) & lives(charles) &
     (killed(agatha,agatha) | killed(butler,agatha) |
      killed(charles,agatha)) &
     (forall(X,forall(Y,
       killed(X,Y) --> hates(X,Y) & ~(richer(X,Y))))) &
     (forall(X,hates(agatha,X)
               --> ~(hates(charles,X)))) &
     (hates(agatha,agatha) & hates(agatha,charles)) &
     (forall(X,lives(X) & ~(richer(X,agatha))
               --> hates(butler,X))) &
     (forall(X,hates(agatha,X) --> hates(butler,X))) &
     (forall(X,~(hates(X,agatha)) | ~(hates(X,butler))
               | ~(hates(X,charles))))
     --> killed(X,agatha)";;
  P55' : int * (string * term) list = 6, ["X", `agatha`]
\end{verbatim}\end{boxed}

\section*{Further reading}

Symbolic differentiation is a classic application of functional languages.
Other symbolic operations are major research issues in their own right ---
\citeN{davenport-ca} give a readable overview of what computer algebra systems
can do and how they work. \citeN{paulson-rewriting} discusses the kind of
simplification strategy we use here in a much more general setting.

Parsing with higher order functions is another popular example. It seems to
have existed in the functional programming folklore for a long time; an early
treatment is given by \citeN{burge-recursive}. Our presentation has been
influenced by the treatments in \citeN{paulson-ml} and \citeN{reade-efp}.

The first definition of `computable real number' in our sense was in the
original paper of \citeN{turing}. His approach was based on decimal expansions,
but he needed to modify it \cite{turing2} for reasons explored in the exercises
below. The approach to real arithmetic given here follows the work of
\citeN{boehm-cartwright}. More recently a high-performance implementation in
CAML Light has been written by \citeN{menisse-thesis}, whose thesis (in French)
contains detailed proofs of correctness for all the algorithms for elementary
transcendental functions. For an alternative approach using `linear fractional
transformations', see \citeN{potts-thesis}.

Our approach to Prolog search and backtracking, either using lists or
continuations, is fairly standard. For more on implementing Prolog, see for
example \citeN{boizumault-prolog}, while for more on the actual use of the
language, the classic text by \citeN{clocksin-mellish} is recommended. A
detailed discussion of continuations in their many guises is given by
\citeN{reynolds-continuations}. The unification algorithm given here is simple
and purely functional, but rather inefficient. For faster imperative
algorithms, see \citeN{martelli-montanari}. Our theorem prover is based on
\mbox{{\sf lean}$T^{\!\!\textstyle A}\!\!P$} \cite{beckert-posegga}. For
another important theorem proving method conveniently based on Prolog-style
search, look at the Prolog Technology Theorem Prover \cite{stickel-pttp}.

\section*{Exercises}

\begin{enumerate}

\item Modify the printer so that each operator has an associativity, used when
printing iterated cases of the operator to avoid excessive parentheses.

\item The differentiation rules for inverses $1 / g(x)$ and quotients $f(x) /
g(x)$ are not valid when $g(x) = 0$. Several others for functions like $ln$ are
also true only under certain conditions. Actually most commercial computer
algebra systems ignore this fact. However, try to do better, by writing a new
version of the differentiator that not only produces the derivative but also a
list of conditions that must be assumed for the derivative to be valid.

\item Try programming a simple procedure for indefinite integration. In
general, it will not be able to do every integral, but try to make it
understand a few basic principles.\footnote{In fact there are algorithms that
can integrate all algebraic expressions (not involving $sin$, $ln$ etc.) ---
see \citeN{davenport-integration}.}

\item Read the documentation for the {\tt format} library, and try to implement
a printer that behaves better when the output is longer than one line.

\item What happens if the parser functions like {\tt term}, {\tt atom} etc. are
all eta-reduced by deleting the word {\tt input}? What happens if
{\tt precedence} is consistently eta-reduced by deleting {\tt input}?

\item How well do precedence parsers generated by {\tt precedence} treat
distinct operators with the same precedence? How can the behaviour be improved?
Can you extend it to allow a mix of left and right associative operators? For
example, one really wants subtraction to associate to the left.

\item Rewrite the parser so that it gives helpful error messages when parsing
fails.

\item Try representing a real number by the function that generates the first
$n$ digits in a decimal expansion. Can you implement the addition function?

\item (*) How can you retain a positional representation while making the basic
operations computable?

\item Implement multiprecision integer arithmetic operations yourself. Make
your multiplication algorithm $O(n^{log_2 3})$ where $n$ is the number of
digits in the arguments.

\item Using {\tt memo} as we have defined it, is it the case that whatever
series of arguments a function is applied to, it will always give the same
answer for the same argument? That is, are the functions true mathematical
functions despite the internal use of references?

\item Write a parser-evaluator to read real-valued expressions and create the
functional representation for the answer.

\item (*) Extend the real arithmetic routines to some functions like $exp$ and
$sin$.

\item Our preprocessing and negation normal form process can take exponential
time in the length of the input in the case of many `if and only if's. Modify
it so that it is linear in this number.

\item Extend the initial transformation into negation normal form so that it
gives Skolem normal form.\footnote{Skolem normal form is covered in most
elementary logic books, e.g. \citeN{enderton-logic}.}

\item (*) Implement a more efficient unification algorithm, e.g. following
\citeN{martelli-montanari}.

\item (*) Implement a version of the Prolog Technology Theorem Prover
\cite{stickel-pttp}

\end{enumerate}

%%% Local Variables:
%%% TeX-master: "all"
%%% End:
