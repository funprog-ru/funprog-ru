\documentstyle[portrait,%
              fancybox,%
              notes,%
              epsfig,%
              alltt,%
              semcolor,
              alltt]{seminar}

\input amssym.def
\input amssym

\newcommand{\nat}{\mbox{$\protect\Bbb N$}}
\newcommand{\num}{\mbox{$\protect\Bbb Z$}}
\newcommand{\rat}{\mbox{$\protect\Bbb Q$}}
\newcommand{\real}{\mbox{$\protect\Bbb R$}}
\newcommand{\complex}{\mbox{$\protect\Bbb C$}}
\newcommand{\xxx}{\mbox{$\protect\Bbb X$}}

\newcommand{\lamb}[1]{\lambda #1.\:}
\newcommand{\eps}[1]{\varepsilon #1.\:}
\newcommand{\all}[1]{\forall #1.\:}
\newcommand{\ex}[1]{\exists #1.\:}
\newcommand{\exu}[1]{\exists! #1.\:}

\newcommand{\True}{\top}
\newcommand{\False}{\bot}
\newcommand{\Not}{\neg}
\newcommand{\And}{\wedge}
\newcommand{\Or}{\vee}
\newcommand{\Imp}{\Rightarrow}
\newcommand{\Iff}{\Leftrightarrow}

\newcommand{\entails}{\vdash}
\newcommand{\proves}{\vdash}

\newcommand{\Ands}{\bigwedge}
\newcommand{\Ors}{\bigvee}

\newcommand{\BQ}{\mbox{$\ulcorner$}}
\newcommand{\BEQ}{\mbox{\raise4pt\hbox{$\ulcorner$}}}
\newcommand{\EQ}{\mbox{$\urcorner$}}
\newcommand{\EEQ}{\mbox{\raise4pt\hbox{$\urcorner$}}}

\newcommand{\QUOTE}[1]{\mbox{$\BQ #1 \EQ$}}
\let\psubset=\subset                    % Pure TeX: thanks to MAJ %
\let\subset=\subseteq

\newcommand{\powerset}{\wp}             % This is pretty dire...  %

\newcommand{\Union}{\cup}
\newcommand{\Inter}{\cap}
\newcommand{\Unions}{\bigcup}
\newcommand{\Inters}{\bigcap}

\newcommand{\proof}{{\bf \noindent Proof:\ }}
\newcommand{\qed}{Q.E.D.}

\newcommand{\Rule}{\infer}

\newcommand{\restrict}{\upharpoonright} % This is lousy and must be fixed! %

\newcommand{\bigsqcap}{\mbox{\Large{$\sqcap$}}}

\newcommand\leb{\lbrack\!\lbrack}
\newcommand\reb{\rbrack\!\rbrack}
\newcommand{\sem}[1]{\leb #1 \reb}

\newcommand{\BA}{\begin{array}[t]{l}}
\newcommand{\EA}{\end{array}}
\newcommand{\sqle}{\sqsubseteq}
\newcommand{\sqlt}{\sqsubset}

\newcommand{\too}{\twoheadrightarrow}

\newcommand{\Los}{{\L}o{\'s}}

% These are from Mike's notes

\def\alphas{\mathrel{\mathop{\longrightarrow}\limits_{\alpha}}}
\def\betas{\mathrel{\mathop{\longrightarrow}\limits_{\beta}}}
\def\etas{\mathrel{\mathop{\longrightarrow}\limits_{\eta}}}

\def\goesto{\longrightarrow}

\newcommand{\defeq}{\stackrel{\bigtriangleup}{=}}

% Sizes

\renewcommand{\slidetopmargin}{0.8in}
\renewcommand{\slidebottommargin}{0.8in}

% Various combinations of one colour on another

\newcommand{\greenonred}[1]%
{\psset{fillcolor=red}\psframebox*[framearc=.3]{\green #1}}

\newcommand{\whiteonred}[1]%
{\psset{fillcolor=red}\psframebox*[framearc=.3]{\white #1}}

\newcommand{\yellowonmagenta}[1]%
{\psset{fillcolor=magenta}\psframebox*[framearc=.3]{\yellow #1}}

\newcommand{\whiteonblack}[1]%
{\psset{fillcolor=black}\psframebox*[framearc=.3]{\white #1}}

\newcommand{\blackonlightgray}[1]%
{\psset{fillcolor=lightgray}\psframebox*[framearc=.3]{\black #1}}

\newcommand{\blueonlightgray}[1]%
{\psset{fillcolor=lightgray}\psframebox*[framearc=.3]{\blue #1}}

\newcommand{\cyanonblack}[1]%
{\psset{fillcolor=black}\psframebox*[framearc=.3]{\cyan #1}}

\newcommand{\blueonyellow}[1]%
{\psset{fillcolor=yellow}\psframebox*[framearc=.3]{\blue #1}}

\newcommand{\redonyellow}[1]%
{\psset{fillcolor=yellow}\psframebox*[framearc=.3]{\red #1}}

\newcommand{\heading}[1]%
{\begin{center}\whiteonblack{\large\bf\blueonlightgray{#1}}\end{center}}

\newcommand{\emphatic}[1]{\blueonyellow{#1}}

\newcommand{\veryemphatic}[1]%
{\begin{center}{\emphatic{#1}}\end{center}}

% Head and foot of slides

\newpagestyle{ColourDemo}%
  {\cyanonblack{Introduction to Functional Programming:
                Lecture 7}\hfil\cyanonblack{\thepage}}
  {\cyanonblack{John Harrison}\hfil
   \cyanonblack{University of Cambridge, 30 January 1997}}

\pagestyle{ColourDemo}

\centerslidesfalse

% Colour bullets

\def\labelitemi{{\black$\bullet$}}
\def\labelitemii{{\black--}}
\def\labelitemiii{{\black$\ast$}}
\def\labelitemiv{{\black$\cdot$}}

% Start of document (default text colour is blue)

\begin{document}\blue


\begin{slide*}

\heading{%
\begin{tabular}{c}
{\LARGE\red Introduction to}\\
{\LARGE\red Functional Programming}\\
{\cyan John Harrison}\\
{\cyan University of Cambridge}\\
{\green Lecture 7}\\
{\green Proving Programs Correct}
\end{tabular}}

\vspace*{0.5cm}

Topics covered:

\begin{itemize}

\item The correctness problem

\item Testing and verification

\item Termination and totality

\item Exponential and gcd

\item Appending and reversing

\end{itemize}

\end{slide*}




\begin{slide*}

\heading{The correctness problem}

\vspace*{0.5cm}

Programs are written to perform some particular task.

However, it is often very hard to write a program that performs its intended
function --- as programmers know well.

In practice, most large programs have `bugs'.

Some bugs are harmless, others merely irritating.

They can cause financial and public relations disasters (e.g. the Pentium FDIV
bug).

In some situation bugs can be deadly.

{\green Peter Neumann: `Computer Related Risks'.}


\end{slide*}



\begin{slide*}

\heading{Dangerous bugs}

\vspace*{0.5cm}

Some situations where bugs can be deadly include:

\begin{itemize}

\item Heart pacemakers

\item Aircraft autopilots

\item Car engine management systems and antilock braking systems

\item Radiation therapy machines

\item Nuclear reactor controllers

\end{itemize}

These applications are said to be {\em safety critical}.

\end{slide*}



\begin{slide*}

\heading{Testing and verification}

\vspace*{0.2cm}

One good way to track down bugs is through extensive testing.

But usually there are too many possible situations to try them all
exhaustively, so there may still be bugs lying undetected.

Program testing can be very useful for demonstrating the presence of bugs, but
it is only in a few unusual cases where it can demonstrate their absence.

An alternative is {\em verification}, where we try to {\em prove} that a
program behaves as required.

Consider ordinary mathematical theorems, like
{\red $$ \Sigma_{n=0}^{n=N} n = \frac{N (N + 1)}{2} $$}
We can {\em test} this for many particular values of $N$, but it is easier and
more satisfactory simply to {\em prove} it (e.g. by induction).

\end{slide*}



\begin{slide*}

\heading{The limits of verification}

\vspace*{0.5cm}

The enterprise of verification can be represented by this diagram:

{\green
\bigskip
\begin{picture}(140,140)(0,0)
\put(40,0){\dashbox(150,25){Actual system}}
\put(40,40){\framebox(150,25){Mathematical model}}
\put(40,80){\framebox(150,25){Mathematical specification}}
\put(40,120){\dashbox(150,25){Actual requirements}}
\put(115,25){\vector(0,1){15}}
\put(115,65){\thicklines \vector(0,1){15}}
\put(115,105){\vector(0,1){15}}
\end{picture}}
\bigskip

It is only the central link that is mathematically precise. The others are
still informal --- all we can do is try to keep them small.

\end{slide*}



\begin{slide*}

\heading{Verifying functional programs}

\vspace*{0.5cm}

We suggested earlier that functional programs might be easier to reason about
formally, because they correspond directly to the mathematical functions that
they represent.

This is arguable, but at least we will try to show that reasoning about some
simple functional programs is straightforward.

We need to remember that, in general, functional programs are {\em partial}
functions. Sometimes we need a separate argument to establish termination.

Often, the proofs proceed by induction, parallelling the definition of the
functions involved by recursion.

\end{slide*}



\begin{slide*}

\heading{Exponentiation (1)}

\vspace*{0.5cm}

Recall the following simple definition of natural number exponentiation:

\begin{black}\begin{verbatim}
   #let rec exp x n =
     if n = 0 then 1
     else x * exp x (n - 1);;
\end{verbatim}\end{black}

We will prove that this satisfies the following specification:

{\red For all $n \geq 0$ and $x$, $\mbox{exp}\; x\; n$ terminates and
$\mbox{exp}\; x\; n = x^n$}

The function is defined by (primitive) recursion.

The proof is by (step-by-step, mathematical) induction.

\end{slide*}



\begin{slide*}

\heading{Exponentiation (2)}

\vspace*{0.5cm}

\begin{itemize}

\item If {\red $n = 0$}, then by definition {\red $\mbox{exp}\; x\; n = 1$}.
Since for any integer {\red $x$}, we have {\red $x^0 = 1$}, so the desired fact
is established.

\item Suppose we know {\red $\mbox{exp}\; x\; n = x^n$}. Because {\red $n \geq
0$}, we also know {\red $n + 1 \not= 0$}. Therefore:
{\red
\begin{eqnarray*}
\mbox{exp}\; x\; (n + 1)   & = & x * \mbox{exp}\; x\; ((n + 1) - 1)     \\
                           & = & x * \mbox{exp}\; x\; n                 \\
                           & = & x * x^n                                \\
                           & = & x^{n + 1}
\end{eqnarray*}}
\end{itemize}
\qed

Note that we assume {\red $0^0 = 1$}, an example of how one must state the
specification precisely!


\end{slide*}



\begin{slide*}

\heading{Greatest common divisor (1)}

\vspace*{0.5cm}

We define a function to calculate the gcd of two integers using Euclid's
algorithm.

\begin{black}\begin{verbatim}
  #let rec gcd x y =
       if y = 0 then x
       else gcd y (x mod y);;
\end{verbatim}\end{black}

We want to prove:

{\red For any integers $x$ and $y$, $\mbox{gcd}\; x\; y$ terminates and returns
a gcd of $x$ and $y$.}

Here we need to be even more careful about the specification. What is a gcd of
two negative numbers?

\end{slide*}




\begin{slide*}

\heading{Greatest common divisor (2)}

\vspace*{0.5cm}

We write {\red $x | y$}, pronounced `{\red $x$} divides {\red $y$}', to mean
that {\red $y$} is an integral multiple of {\red $x$}, i.e. there is some
integer {\red $d$} with {\red $y = d x$}.

We say that {\red $d$} is a {\em common divisor} of {\red $x$} and {\red $y$}
if {\red $d | x$} and {\red $d | y$}.

We say that {\red $d$} is a {\em greatest} common divisor if:

\begin{itemize}

\item We have {\red $d | x$} and {\red $d | y$}

\item For any other integer {\red $d'$}, if {\red $d' | x$} and {\red $d' | y$}
then {\red $d' | d$}.

\end{itemize}

Note that unless {\red $x$} and {\red $y$} are both zero, we do not specify the
sign of the gcd. The specification does not constrain the implementation
completely.

\end{slide*}




\begin{slide*}

\heading{Greatest common divisor (3)}

\vspace*{0.5cm}

Now we come to the proof. The {\black \tt gcd} function is no longer defined by
{\em primitive} recursion.

In fact, {\black \tt gcd x y} is defined in terms of {\black \tt gcd y (x mod
y)} in the step case.

We do not, therefore, proceed by step-by-step mathematical induction, but by
{\em wellfounded} induction on {\red $|y|$}.

The idea is that this quantity (often called a {\em measure}) decreases with
each call. We can use it to prove termination, and as a handle for wellfounded
induction.

In complicated recursions, finding the right wellfounded ordering on the
arguments can be tricky. But in many cases one can use this simple `measure'
approach.

\end{slide*}



\begin{slide*}

\heading{Greatest common divisor (4)}

\vspace*{0.5cm}

Now we come to the proof. Fix some arbitrary {\red $n$}. We suppose that
the theorem is established for all arguments {\red $x$} and {\red
$y$} with {\red $|y| < n$}, and we try to prove it for all {\red
$x$} and {\red $y$} with {\red $|y| = n$}. There are
two cases.

First, suppose that {\red $y = 0$}. Then {\red $\mbox{gcd}\; x\; y
= x$} by definition. Now trivially {\red $x | x$} and {\red $x |
0$}, so it is a common divisor.

Suppose {\red $d$} is another common divisor, i.e. {\red $d | x$}
and {\red $d | 0$}. Then immediately we get {\red $d | x$}, so
$x$ is a {\em greatest} common divisor.

This establishes the first part of the induction proof.

\end{slide*}





\begin{slide*}

\heading{Greatest common divisor (5)}

\vspace*{0.5cm}

Now suppose {\red $y \not= 0$}. We want to apply the inductive hypothesis to
{\red $\mbox{gcd}\; y\; (x\; \mbox{mod}\; y)$}.

We will write {\red $r = x\; \mbox{mod}\; y$} for short. The basic property of
the {\tt mod} function that we use is that, since {\red $y \not= 0$}, for some
integer {\red $q$} we have {\red $x = q y + r$} and {\red $|r| < |y|$}.

Since {\red $|r| < |y|$}, the inductive hypothesis tells us that {\red $d =
\mbox{gcd}\; y\; (x\; \mbox{mod}\; y)$} is a gcd of {\red $y$} and {\red $r$}.

We just need to show that it is a gcd of {\red $x$} and {\red $y$}. It is
certainly a common divisor, since if {\red $d | y$} and {\red $d | r$} we have
{\red $d | x$}, as {\red $x = q y + r$}.

Now suppose {\red $d' | x$} and {\red $d' | y$}. By the same equation, we find
that {\red $d' | r$}. Thus {\red $d'$} is a common divisor of {\red $y$} and
{\red $r$}, but then by the inductive hypothesis, {\red $d' | d$} as required.

\end{slide*}






\begin{slide*}

\heading{Append (1)}

\vspace*{0.5cm}

Now consider an example concerning lists rather than numbers. Define:

\begin{black}\begin{verbatim}
  #let rec append l1 l2 =
     match l1 with
       [] -> l2
     | (h::t) -> h::(append t l2);;
\end{verbatim}\end{black}

This is supposed to join together two lists. We want to prove that the
operation is associative, i.e. for any three lists {\red $l_1$}, {\red $l_2$}
and {\red $l_3$} we have:
{\red $$ \mbox{append}\; l_1\; (\mbox{append}\; l_2\; l_3) = \mbox{append}\;
(\mbox{append}\; l_1\; l_2)\; l_3 $$}
We can proceed by induction on the length of {\red $l_1$}, but since the
function was defined by structural recursion over lists, it is more natural to
prove the theorem by {\em structural induction}.

The principle is: if a property holds for the empty list, and whenever it holds
for {\red $t$} it holds for any {\red $h::t$}, then it holds for any list.

\end{slide*}






\begin{slide*}

\heading{Append (2)}

\vspace*{0.5cm}

We proceed, then, by structural induction on {\red $l_1$}. There are two cases
to consider. First, suppose {\red $l_1 = []$}. Then we have:

{\red \begin{eqnarray*}
& & \mbox{append}\; l_1\; (\mbox{append}\; l_2\; l_3)           \\
& = & \mbox{append}\; []\; (\mbox{append}\; l_2\; l_3)          \\
& = & \mbox{append}\; l_2\; l_3                                 \\
& = & \mbox{append}\; (\mbox{append}\; []\; l_2)\; l_3          \\
& = & \mbox{append}\; (\mbox{append}\; l_1\; l_2)\; l_3
\end{eqnarray*}}

As required.

\end{slide*}


\begin{slide*}

\heading{Append (3)}

\vspace*{0.5cm}

Now suppose {\red $l_1 = h::t$}. We may assume that for any {\red $l_2$}
and {\red $l_3$} we have:
{\red $$ \mbox{append}\; t\; (\mbox{append}\; l_2\; l_3) = \mbox{append}\;
(\mbox{append}\; t\; l_2)\; l_3 $$}
\noindent Therefore:
{\red \begin{eqnarray*}
& & \mbox{append}\; l_1\; (\mbox{append}\; l_2\; l_3)           \\
& = & \mbox{append}\; (h::t)\; (\mbox{append}\; l_2\; l_3)      \\
& = & h::(\mbox{append}\; t\; (\mbox{append}\; l_2\; l_3))      \\
& = & h::(\mbox{append}\; (\mbox{append}\; t\; l_2)\; l_3)      \\
& = & \mbox{append}\; (h::(\mbox{append}\; t\; l_2))\; l_3)     \\
& = & \mbox{append}\; (\mbox{append}\; (h::t)\; l_2)\; l_3)     \\
& = & \mbox{append}\; (\mbox{append}\; l_1\; l_2)\; l_3)
\end{eqnarray*}}
The theorem is proved.

\end{slide*}







\begin{slide*}

\heading{Reverse (1)}

\vspace*{0.2cm}

For a final example, let us define a function to reverse a list:
\begin{black}\begin{verbatim}
  #let rec rev =
     fun [] -> []
       | (h::t) -> append (rev t) [h];;
  rev : 'a list -> 'a list = <fun>
  #rev [1;2;3];;
  - : int list = [3; 2; 1]
\end{verbatim}\end{black}
We will prove that for any list {\red $l$} we have:
{\red $$ \mbox{rev} (\mbox{rev}\; l) = l $$}
This is again a structural induction. However we require two lemmas, which can
also be proved by structural induction (details in the notes):
{\red
\begin{eqnarray*}
\mbox{append}\; l\; [] & = & l          \\
\mbox{rev}(\mbox{append}\; l_1\; l_2) & = & \mbox{append}\; (\mbox{rev}\; l_2)\; (\mbox{rev}\; l_1)
\end{eqnarray*}}
\end{slide*}



\begin{slide*}

\heading{Reverse (2)}

\vspace*{0.5cm}

First suppose that {\red $l = []$}. Then the proof is easy:

{\red
\begin{eqnarray*}
\mbox{rev}(\mbox{rev}\; l)
& = & \mbox{rev}(\mbox{rev}\; [])                               \\
& = & \mbox{rev}\; []                                               \\
& = & []                                                                \\
& = & l
\end{eqnarray*}
}

Now suppose that {\red $l = h::t$} and we know that

{\red $$ \mbox{rev}(\mbox{rev}\; t) = t $$}

\end{slide*}

\begin{slide*}

\heading{Reverse (3)}

\vspace*{0.5cm}

{\red
\begin{eqnarray*}
& & \mbox{rev}(\mbox{rev}\; l)                                  \\
& = & \mbox{rev}(\mbox{rev}\; (h::t))                           \\
& = & \mbox{rev}(\mbox{append}\; (\mbox{rev}\; t)\; [h])        \\
& = & \mbox{append}\; (\mbox{rev}\; [h])\; (\mbox{rev} (\mbox{rev}\; t)) \\
& = & \mbox{append}\; (\mbox{rev}\; [h])\; t                        \\
& = & \mbox{append}\; (\mbox{rev}\; (h::[]))\; t                    \\
& = & \mbox{append}\; (\mbox{append}\; []\; [h])\; t                    \\
& = & \mbox{append}\; [h]\; t                                           \\
& = & \mbox{append}\; (h::[])\; t                                       \\
& = & h::(\mbox{append}\; []\; t)                                       \\
& = & h::t                                                              \\
& = & l
\end{eqnarray*}}

\end{slide*}

\end{document}

%%% Local Variables:
%%% TeX-master: "all"
%%% End:
