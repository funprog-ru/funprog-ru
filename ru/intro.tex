\chapter{Introduction}

Programs in traditional languages, such as FORTRAN, Algol, C and Modula-3, rely
heavily on modifying the values of a collection of variables, called the {\em
state}. If we neglect the input-output operations and the possibility that a
program might run continuously (e.g. the controller for a manufacturing
process), we can arrive at the following abstraction. Before execution, the
state has some initial value $\sigma$, representing the inputs to the program,
and when the program has finished, the state has a new value $\sigma'$
including the result(s). Moreover, during execution, each command changes the
state, which has therefore proceeded through some finite sequence of values:

$$ \sigma = \sigma_0 \to \sigma_1 \to \sigma_2 \to \cdots \to \sigma_n =
\sigma' $$

For example in a sorting program, the state initially includes an array of
values, and when the program has finished, the state has been modified in such
a way that these values are sorted, while the intermediate states represent
progress towards this goal.

The state is typically modified by {\em assignment} commands, often written in
the form {\tt v = E} or {\tt v := E} where {\tt v} is a variable and {\tt E}
some expression. These commands can be executed in a sequential manner by
writing them one after the other in the program, often separated by a
semicolon. By using statements like {\tt if} and {\tt while}, one can execute
these commands conditionally, and repeatedly, depending on other properties of
the current state. The program amounts to a set of instructions on how to
perform these state changes, and therefore this style of programming is often
called {\em imperative} or {\em procedural}. Correspondingly, the traditional
languages intended to support it are known as imperative or procedural
languages.

Functional programming represents a radical departure from this model.
Essentially, a functional program is simply an expression, and execution means
evaluation of the expression.\footnote{Functional programming is often called
`applicative programming' since the basic mechanism is the {\em application} of
functions to arguments.} We can see how this might be possible, in general
terms, as follows. Assuming that an imperative program (as a whole) is
deterministic, i.e. the output is completely determined by the input, we can
say that the final state, or whichever fragments of it are of interest, is some
function of the initial state, say $\sigma' = f(\sigma)$.\footnote{Compare
Naur's remarks \cite{raphael-spl} that he can write any program in a single
statement $Output = Program(Input)$.} In functional programming this view is
emphasized: the program is actually an expression that corresponds to the
mathematical function $f$. Functional languages support the construction of
such expressions by allowing rather powerful functional constructs.

Functional programming can be contrasted with imperative programming either
in a negative or a positive sense. Negatively, functional programs do not use
variables --- there {\em is} no state. Consequently, they cannot use
assignments, since there is nothing to assign to. Furthermore the idea of
executing commands in sequence is meaningless, since the first command can make
no difference to the second, there being no state to mediate between them.
Positively however, functional programs can use functions in much more
sophisticated ways. Functions can be treated in exactly the same way as simpler
objects like integers: they can be passed to other functions as arguments and
returned as results, and in general calculated with. Instead of sequencing and
looping, functional languages use recursive functions, i.e. functions that are
defined in terms of themselves. By contrast, most traditional languages provide
poor facilities in these areas. C allows some limited manipulation of functions
via pointers, but does not allow one to create new functions dynamically.
FORTRAN does not even support recursion at all.

To illustrate the distinction between imperative and functional programming,
the factorial function might be coded imperatively in C (without using C's
unusual assignment operations) as:

\begin{lstlisting}[language=C]
  int fact(int n)
  { int x = 1;
    while (n > 0)
     { x = x * n;
       n = n - 1;
     }
    return x;
  }
\end{lstlisting}

\noindent whereas it would be expressed in ML, the functional language we
discuss later, as a recursive function:

\begin{lstlisting}
  let rec fact n =
    if n = 0 then 1
    else n * fact(n - 1);;
\end{lstlisting}

In fact, this sort of definition can be used in C too. However for more
sophisticated uses of functions, functional languages stand in a class by
themselves.

\section{The merits of functional programming}

At first sight, a language without variables or sequencing might seem
completely impractical. This impression cannot be dispelled simply by a few
words here. But we hope that by studying the material that follows, readers
will gain an appreciation of how it is possible to do a lot of interesting
programming in the functional manner.

There is nothing sacred about the imperative style, familiar though it is. Many
features of imperative languages have arisen by a process of abstraction from
typical computer hardware, from machine code to assemblers, to macro
assemblers, and then to FORTRAN and beyond. There is no reason to suppose that
such languages represent the most palatable way for humans to communicate
programs to a machine. After all, existing hardware designs are not sacred
either, and computers are supposed to do our bidding rather than conversely.
Perhaps the right approach is not to start from the hardware and work upwards,
but to start with programming languages as an abstract notation for specifying
algorithms, and then work {\em down} to the hardware
\cite{dijkstra-discipline}. Actually, this tendency can be detected in
traditional languages too. Even FORTRAN allows arithmetical expressions to be
written in the usual way. The programmer is not burdened with the task of
linearizing the evaluation of subexpressions and finding temporary storage for
intermediate results.

This suggests that the idea of developing programming languages quite different
from the traditional imperative ones is certainly defensible. However, to
emphasize that we are not merely proposing change for change's sake, we should
say a few words about why we might prefer functional programs to imperative
ones.

Perhaps the main reason is that functional programs correspond more directly to
mathematical objects, and it is therefore easier to reason about them. In order
to get a firm grip on exactly what programs mean, we might wish to assign an
abstract mathematical meaning to a program or command --- this is the aim of
{\em denotational semantics} (semantics = meaning). In imperative languages,
this has to be done in a rather indirect way, because of the implicit
dependency on the value of the state. In simple imperative languages, one can
associate a command with a function $\Sigma \to \Sigma$, where $\Sigma$ is the
set of possible values for the state. That is, a command takes some state and
produces another state. It may fail to terminate (e.g. {\tt while true do x :=
x}), so this function may in general be partial. Alternative semantics are
sometimes preferred, e.g. in terms of {\em predicate transformers}
\cite{dijkstra-discipline}. But if we add features that can pervert the
execution sequence in more complex ways, e.g. {\tt goto}, or C's {\tt break}
and {\tt continue}, even these interpretations no longer work, since one
command can cause the later commands to be skipped. Instead, one typically uses
a more complicated semantics based on {\em continuations}.

By contrast functional programs, in the words of \citeN{henson-book}, `wear
their semantics on their sleeves'.\footnote{More: denotational semantics can be
seen as an attempt to turn imperative languages into functional ones by making
the state explicit.} We can illustrate this using ML. The basic datatypes have
a direct interpretation as mathematical objects. Using the standard notation of
$\sem{X}$ for `the semantics of $X$', we can say for example that
$\sem{\mbox{int}} = \num$. Now the ML function {\tt fact} defined by:

\begin{lstlisting}
  let rec fact n =
    if n = 0 then 1
    else n * fact(n - 1);;
\end{lstlisting}

\noindent has one argument of type {\tt int}, and returns a value of type {\tt
int}, so it can simply be associated with an abstract partial function
$\num \to \num$:

$$ \sem{\mbox{fact}}(n) = \left\{ \begin{array}{ll}
                                 n! & \mbox{if $n \geq 0$} \\
                                 \bot & \mbox{otherwise}
                          \end{array} \right. $$

(Here $\bot$ denotes undefinedness, since for negative arguments, the program
fails to terminate.) This kind of simple interpretation, however, fails in
non-functional programs, since so-called `functions' might not be functions at
all in the mathematical sense. For example, the standard C library features a
function {\tt rand()}, which returns different, pseudo-random values on
successive calls. This sort of thing can be implemented by using a local static
variable to remember the previous result, e.g:

\begin{lstlisting}[language=C]
  int rand(void)
  { static int n = 0;
    return n = 2147001325 * n + 715136305;
  }
\end{lstlisting}

Thus, one can see the abandonment of variables and assignments as the logical
next step after the abandonment of {\tt goto}, since each step makes the
semantics simpler. A simpler semantics makes reasoning about programs more
straightforward. This opens up more possibilities for correctness proofs, and
for provably correct transformations into more efficient programs.

Another potential advantage of functional languages is the following. Since the
evaluation of expressions has no side-effect on any state, separate
subexpressions can be evaluated in any order without affecting each other. This
means that functional programs may lend themselves well to parallel
implementation, i.e. the computer can automatically farm out different
subexpressions to different processors. By contrast, imperative programs often
impose a fairly rigid order of execution, and even the limited interleaving of
instructions in modern pipelined processors turns out to be complicated and
full of technical problems.

Actually, ML is not a purely functional programming language; it does have
variables and assignments if required. Most of the time, we will work inside
the purely functional subset. But even if we do use assignments, and lose some
of the preceding benefits, there are advantages too in the more flexible use of
functions that languages like ML allow. Programs can often be expressed in a
very concise and elegant style using higher-order functions (functions that
operate on other functions).\footnote{Elegance is subjective and conciseness is
not an end in itself. Functional languages, and other languages like APL, often
create a temptation to produce very short tricky code which is elegant to
cognoscenti but obscure to outsiders.} Code can be made more general, since it
can be parametrized even over other functions. For example, a program to add up
a list of numbers and a program to multiply a list of numbers can be seen as
instances of the same program, parametrized by the pairwise arithmetic
operation and the corresponding identity. In one case it is given $+$ and $0$
and in the other case, $*$ and $1$.\footnote{This parallels the notion of
abstraction in pure mathematics, e.g. that the additive and multiplicative
structures over numbers are instances of the abstract notion of a monoid. This
similarly avoids duplication and increases elegance.} Finally, functions can
also be used to represent {\em infinite} data in a convenient way --- for
example we will show later how to use functions to perform exact calculation
with real numbers, as distinct from floating point approximations.

At the same time, functional programs are not without their problems. Since
they correspond less directly the the eventual execution in hardware, it can be
difficult to reason about their exact usage of resources such as time and
space. Input-output is also difficult to incorporate neatly into a functional
model, though there are ingenious techniques based on infinite sequences.

It is up to readers to decide, after reading this book, on the merits of the
functional style. We do not wish to enforce any ideologies, merely to point out
that there {\em are} different ways of looking at programming, and that in the
right situations, functional programming may have considerable merits. Most of
our examples are chosen from areas that might loosely be described as `symbolic
computation', for we believe that functional programs work well in such
applications. However, as always one should select the most appropriate tool
for the job. It may be that imperative programming, object-oriented programming
or logic programming are more suited to certain tasks. Horses for courses.

\section{Outline}

For those used to imperative programming, the transition to functional
programming is inevitably difficult, whatever approach is taken. While some
will be impatient to get quickly to real programming, we have chosen to start
with lambda calculus, and show how it can be seen as the theoretical
underpinning for functional languages. This has the merit of corresponding
quite well to the actual historical line of development.

So first we introduce lambda calculus, and show how what was originally
intended as a formal logical system for mathematics turned out to be a
completely general programming language. We then discuss why we might want to
add types to lambda calculus, and show how it can be done. This leads us into
ML, which is essentially an extended and optimized implementation of typed
lambda calculus with a certain evaluation strategy. We cover the practicalities
of basic functional programming in ML, and discuss polymorphism and most
general types. We then move on to more advanced topics including exceptions and
ML's imperative features. We conclude with some substantial examples, which we
hope provide evidence for the power of ML.

\section*{Further reading}

Numerous textbooks on `functional programming' include a general introduction
to the field and a contrast with imperative programming --- browse through a
few and find one that you like. For example, \citeN{henson-book} contains a
good introductory discussion, and features a similar mixture of theory and
practice to this text. A detailed and polemical advocacy of the functional
style is given by \citeN{backus-liberated}, the main inventor of FORTRAN.
\citeN{gordon-io} discusses the problems of incorporating input-output into
functional languages, and some solutions. Readers interested in denotational
semantics, for imperative and functional languages, may look at
\citeN{winskel-sem}.

%%% Local Variables:
%%% TeX-master: "all"
%%% End:
